\documentclass[12pt,a4paper]{article} 
\usepackage{amsmath,amssymb,amsthm,graphicx,hyperref}
\usepackage[BoldFont,SlantFont]{xeCJK}  
\setCJKmainfont{cwTeX Q Ming Medium}
\defaultCJKfontfeatures{AutoFakeBold=6,AutoFakeSlant=.4} 
\newCJKfontfamily\Kai{cwTeX Q Kai Medium} 
\newCJKfontfamily\Hei{cwTeX Q Hei Bold}
\newCJKfontfamily\Yuan{cwTeX Q Yuan Medium}
\newCJKfontfamily\Ming{cwTeX Q Ming Medium}
\newCJKfontfamily\Song{cwTeX Q Fangsong Medium}
\begin{document}
\title{初步意見}
\author{}
\maketitle

\section{文章敘述部份}

\begin{enumerate}
  \item (2.1)式中 $N_t$ 應為 $N(t)$。
  \item (2.7)式最後一行的無窮級數表示式的推導過程應再交代清楚。
  \item Lemma 2.1 證明引用文獻(5)之結果,應標明出處(pp.13(8))。
  \item Lemma 2.2 敘述與證明過於簡略--可參照文獻(6)之 Proposition 2.1。 
  \item 作者僅以``All processes are assumed under the risk neutral $Q$ on a complete probability space $(\Omega, \mathcal{F}, Q)$'' (pp.3) 帶過測度空間架構的設定,過份簡略。較完整敘述至少應以文獻(6) Proposition 2.1上方為參考,或參照Cox S. et al. ``Valuation of Structured Risk Management Products'', IME 34, 259--272一文的完整寫法。
\end{enumerate}

\section{研究結果部份}
\begin{enumerate}
  \item 本文與作者另篇文章(文獻(12))多有類似,惟本文另外考慮option issuer提早違約風險;數值結果應再包含與文獻(12),未考慮提早違約風險情形之比較。
  \item 是否可修改option issuer dynamic (2.5)式為jump-diffusion?
\end{enumerate}

\end{document}
