\documentclass[a4paper,12pt]{book}
\usepackage{amsmath,amsfonts,amssymb,amsthm,graphicx,hyperref,booktabs}
\usepackage[mathscr]{eucal}
\usepackage[a4paper,inner=1.5cm,outer=3cm,top=2cm,bottom=3cm,bindingoffset=1cm]{geometry}
\usepackage[numbers,sort&compress]{natbib}

\usepackage{listings} 
\lstset{                 
basicstyle=\ttfamily,   % code in typewriter font
numbers=left,           % where to put the line-numbers
numberstyle=\tiny,      % the size of the fonts used for the line-numbers
numbersep=5pt,          % how far the line-numbers are from the code
frame=single,           % adds a frame around the code
breaklines=true,        % sets automatic line breaking
columns=flexible        % important!
}

% 中文配置
\usepackage{xeCJK}
\setCJKmainfont{cwTeX Q Ming Medium}
\defaultCJKfontfeatures{AutoFakeBold=4,AutoFakeSlant=.4} 
\newCJKfontfamily\Kai{cwTeX Q Kai Medium} 
\newCJKfontfamily\Hei{cwTeX Q Hei Bold}
\newCJKfontfamily\Yuan{cwTeX Q Yuan Medium}
\newCJKfontfamily\Ming{cwTeX Q Ming Medium}
\newCJKfontfamily\Song{cwTeX Q Fangsong Medium}

\theoremstyle{definition}
\newtheorem{lmm}{Lemma}
\newtheorem{thm}{Theorem}
\newtheorem{dfn}{Definition}
\newtheorem{cor}{Corollary}
\newtheorem{prp}{Proposition}
\newtheorem{ex}{Example}
\newtheorem{prob}{Problem}

% Definition of various shortcuts.
\DeclareMathOperator\curl{curl}
\DeclareMathOperator\divv{div}
%\DeclareMathOperator\curl{\nabla\times}
%\DeclareMathOperator\divv{\nabla\cdot}
\DeclareMathOperator\Curlv{{\bf curl}_\Gamma}
\DeclareMathOperator\im{\Im}
\DeclareMathOperator\re{\Re}
\DeclareMathOperator\comp{\sf{K}}
\DeclareMathOperator\img{\EuScript{R}}
\DeclareMathOperator\pv{pv}
\DeclareMathOperator\dom{dom}
\DeclareMathOperator\nul{\EuScript{N}}
\DeclareMathOperator\rank{rk}
\DeclareMathOperator\col{col}
\DeclareMathOperator\range{range}
\DeclareMathOperator\dist{dist}
\newcommand\bdr{\Gamma}
\newcommand\Div{\divv_\Gamma}
\newcommand\Curl{\curl_\Gamma}
\newcommand\lTS{{\bf L}^2_\text{t}(\mathbb{S}^2)}
\newcommand\lTD{{\bf H}^{-\frac{1}{2}}(\Div)}
\newcommand\lTC{{\bf H}^{-\frac{1}{2}}(\Curl)}
\newcommand\lT{L_{2,\text{t}}(\bdr)}
\newcommand\nltwo[1]{\|#1\|_{L_2(\bdr)}}
\newcommand\nltwod[1]{\|#1\|_{\lTD}}
\newcommand\nltwoc[1]{\|#1\|_{\lTC}}
\newcommand\intv[1]{\int_{\Omega}#1\,\text{d}V}
\newcommand\ints[1]{\int_{\bdr}#1\,\text{d}\sigma}
\newcommand\intvv[1]{\int_{\Omega}#1}
\newcommand\intss[1]{\int_{\bdr}#1}
\newcommand\prb[1]{\begin{prob}#1\end{prob}}
\newcommand\sol[1]{\begin{proof}[Solution]#1\end{proof}}
  
% Trick for defining 'phantom' 
\newcommand\relphantom[1]{\mathrel{\phantom{#1}}}

% number within chapter: e.g.(1.1)
\numberwithin{equation}{chapter}
\numberwithin{thm}{chapter}
\numberwithin{prp}{chapter}
\numberwithin{lmm}{chapter}
\numberwithin{dfn}{chapter}
\numberwithin{cor}{chapter}

% principal value integral
\def\Xint#1{\mathchoice
  {\XXint\displaystyle\textstyle{#1}} 
  {\XXint\textstyle\scriptstyle{#1}} 
  {\XXint\scriptstyle\scriptscriptstyle{#1}} 
  {\XXint\scriptscriptstyle\scriptscriptstyle{#1}} 
\!\int}

\def\XXint#1#2#3{{\setbox0=\hbox{$#1{#2#3}{\int}$ }
\vcenter{\hbox{$#2#3$ }}\kern-.56\wd0}}
\def\ddashint{\Xint=}
\def\dashint{\Xint-}

% in tables with \toprule, set 12pt vertical space
\setlength{\abovetopsep}{12pt}

\begin{document}

%\author{Chang-ye Tu}
\title{Inverse Obstacle Scattering of a Perfect Conductor}
%\date{\today}
\maketitle

\section{Prerequisites}
\begin{prp}[Green Formula]
  \begin{multline*}
    \intv{\left(E\cdot\Delta H - H\cdot\Delta E\right)}\\=\ints{\left(E\times\curl H + E\divv H - H\times\curl E - H\divv E\right)\cdot\nu}
  \end{multline*}
  If $\divv E=\divv H=0$, then
  \begin{align}\label{eq:green}
    \intv{E&\cdot\curl\curl H - H\cdot\curl\curl E}\\
    &=\ints{\left(E\times\curl H - H\times\curl E\right)\cdot\nu}\\
    &=\ints{(\nu\times E)\cdot\curl H - (\nu\times H)\cdot\curl E}
  \end{align}
\end{prp}

\begin{prp}[Stratton-Chu Representation Formula]\label{prp:chu}
  If $E, H\in C^1(\Omega_+)\cap C(\Omega_+\cup\bdr)$ satisfy Maxwell equations in $\Omega_+$ and the Silver-M\"uller radiation condition, then for $x\in\Omega_+$
  \begin{multline*}
    E(x) = \curl\ints{\nu(x)\times E(y)\Phi(x,y)}(y)\\
           +\frac{i}{k}\curl\curl\ints{\nu(y)\times H(y)\Phi(x,y)}(y)
  \end{multline*}
  \begin{multline*}
    H(x) = \curl\ints{\nu(x)\times H(y)\Phi(x,y)}(y)\\
           -\frac{i}{k}\curl\curl\ints{\nu(y)\times E(y)\Phi(x,y)}(y).
  \end{multline*}
\end{prp}

\begin{prp}[Far Field Patterns]\label{prp:far}
  \begin{align*}
    E^\infty(\hat{x}) &= ik\,\hat{x}\times\ints{\left\{\nu(y)\times E(y)+(\nu(y)\times H(y))\times\hat{x}\right\}e^{-ik\hat{x}\cdot y}}(y) \\
    H^\infty(\hat{x}) &= ik\,\hat{x}\times\ints{\left\{\nu(y)\times H(y)-(\nu(y)\times E(y))\times\hat{x}\right\}e^{-ik\hat{x}\cdot y}}(y) 
  \end{align*}
\end{prp}

\begin{prp}[Rellich Lemma]
  If $E, H\in C^1(\Omega_+)$ is a radiating solution of Maxwell equations such that the electric far field pattern vanishes identically, then $E=H=0$ in $\Omega_+$.
\end{prp}

\section{Reciprocity Relations}

Assume $x, z\in\Omega_+$, $\hat{x}, d\in\mathbb{S}^2$, $p, q\in\mathbb{R}^3$.  

Given the incident electromagentic wave 
\begin{align*}
  E_\text{w}^\text{i}(x, d, p) &= \frac{i}{k}\curl_x\curl_x p e^{ikx\cdot d} = ik(d\times p)\times d e^{ikx\cdot d}, \\
  H_\text{w}^\text{i}(x, d, p) &= \curl_x p e^{ikx\cdot d} = ik(d\times p)e^{ikx\cdot d}, 
\end{align*} 
the scattered field is denoted by 
$$E_\text{w}^\text{s}(x, d, p),\quad H_\text{w}^\text{s}(x, d, p)$$
with corresponding far field pattern
$$E_\text{w}^\infty(\hat{x}, d, p), \quad H_\text{w}^\infty(\hat{x}, d, p).$$

Given the incident dipole 
\begin{align*}
  E_\text{p}^\text{i}(x, z, p) &= \frac{i}{k}\curl_x\curl_x p\Phi(x, z), \\
  H_\text{p}^\text{i}(x, z, p) &= \curl_x p\Phi(x, z),
\end{align*}
the scattered field is denoted by 
$$E_\text{p}^\text{s}(x, z, p),\quad H_\text{p}^\text{s}(x, z, p)$$ 
with the corresponding far field pattern
$$E_\text{p}^\infty(\hat{x}, z, p),\quad H_\text{p}^\infty(\hat{x}, z, p).$$

The total field is denoted by 
\begin{align*}
  E_\text{w}(x, d, p) &= E_\text{w}^\text{i}(x, d, p) + E_\text{w}^\text{s}(x, d, p)\\
  H_\text{w}(x, d, p) &= H_\text{w}^\text{i}(x, d, p) + H_\text{w}^\text{s}(x, d, p) \\
  E_\text{p}(x, z, p) &= E_\text{p}^\text{i}(x, z, p) + E_\text{p}^\text{s}(x, z, p)\\
  H_\text{p}(x, z, p) &= H_\text{p}^\text{i}(x, z, p) + H_\text{p}^\text{s}(x, z, p)
\end{align*}

\begin{thm}[Mixed Reciprocity Relation]
  $$p\cdot E_\text{w}^\text{s}(z, -\hat{x}, q) = 4\pi q\cdot E_\text{p}^\infty(\hat{x}, z, p)$$ 
\end{thm}

\begin{proof}
  From proposition \eqref{prp:far} we have
  \begin{multline}\label{eq:mr1}
    4\pi q\cdot E_\text{p}^\infty(\hat{x}, z, p) = \ints{\nu(y)\times E_\text{p}^\text{s}(y, z, p)\cdot H_\text{w}^\text{i}(y, -\hat{x}, q) \\+ \nu(y)\times H_\text{p}^\text{s}(y, z, p)\cdot E_\text{w}^\text{i}(y, -\hat{x}, q)}(y) 
  \end{multline}
  From Green formula \eqref{eq:green} we have
  \begin{multline}\label{eq:mr2}
    \ints{\nu(y)\times E_\text{p}^\text{s}(y, z, p)\cdot H_\text{w}^\text{s}(y, -\hat{x}, q) \\+ \nu(y)\times H_\text{p}^\text{s}(y, z, p)\cdot E_\text{w}^\text{s}(y, -\hat{x}, q)}(y) = 0 
  \end{multline}
  Add \eqref{eq:mr1}, \eqref{eq:mr2} and apply the boundary condition $$\nu(y)\times E_\text{w}(y, -\hat{x}, q)=0\quad\forall y\in\bdr$$ we have
  \begin{align}\label{eq:mr3}
    4\pi q\cdot E_\text{p}^\infty(\hat{x}, z, p)= \ints{\nu(y)\times E_\text{p}^\text{s}(y, z, p)\cdot H_\text{w}(y, -\hat{x}, q)}(y) 
  \end{align}
  From Stratton-Chu representation,  
  \begin{multline}\label{eq:mr4}
    E_\text{w}^\text{s}(z, -\hat{x}, q) = \curl\ints{\nu(y)\times E_\text{w}^\text{s}(y, -\hat{x}, q)\Phi(z, y)}(y)\\+\frac{i}{k}\curl\curl\ints{\nu(y)\times H_\text{w}^\text{s}(y, -\hat{x}, q)\Phi(z, y)}(y)
  \end{multline}
  From Green formula \eqref{eq:green},
  \begin{multline}\label{eq:mr5}
    0 = \curl\ints{\nu(y)\times E_\text{w}^\text{i}(y, -\hat{x}, q)\Phi(z, y)}(y)\\+\frac{i}{k}\curl\curl\ints{\nu(y)\times H_\text{w}^\text{i}(y, -\hat{x}, q)\Phi(z, y)}(y)
  \end{multline}
  Add \eqref{eq:mr4}, \eqref{eq:mr5} and apply the boundary condition $$\nu(y)\times E_\text{w}(y, -\hat{x}, q)=0\quad\forall y\in\bdr$$ we have
  \begin{align}\label{eq:mr6}
    E_\text{w}^\text{s}(z, -\hat{x}, q) = \frac{i}{k}\curl\curl\ints{\nu(y)\times H_\text{w}(y, -\hat{x}, q)\Phi(z, y)}(y)
  \end{align}
  From \eqref{eq:mr6}, the identity $$p\cdot\curl\curl_z\{a(y)\Phi(z, y)\}=a(y)\cdot\curl\curl_z\{p\Phi(z, y)\},$$ and the boundary condition $$\nu(y)\times E_\text{p}^\text{i}(y, z, p) = -\nu(y)\times E_\text{p}^\text{s}(y, z, p)\quad\forall y\in\bdr$$ we have 
  \begin{align*}
    p\cdot E_\text{w}^\text{s}(z, -\hat{x}, q) &= \frac{i}{k}\,p\cdot\curl\curl\ints{\nu(y)\times H_\text{w}(y, -\hat{x}, q)\Phi(z, y)}(y) \\
    &=\frac{i}{k}\ints{\nu(y)\times H_\text{w}(y, -\hat{x}, q)\cdot\curl\curl\{p\Phi(z, y)\}}(y) \\
    &=\ints{\nu(y)\times H_\text{w}(y, -\hat{x}, q)\cdot E_\text{p}^\text{i}(y, z, p)}(y) \\
    &=-\ints{\nu(y)\times E_\text{p}^\text{i}(y, z, p)\cdot H_\text{w}(y, -\hat{x}, q)}(y) \\
    &=\ints{\nu(y)\times E_\text{p}^\text{s}(y, z, p)\cdot H_\text{w}(y, -\hat{x}, q)}(y),
   \end{align*}
   which equals \eqref{eq:mr3}.
\end{proof}

\begin{thm}[Reciprocity Relation]
  $$q\cdot E_\text{w}^\infty(\hat{x}, d, p) = p\cdot E_\text{w}^\infty(-d, -\hat{x}, q)$$ 
\end{thm}

\begin{proof}
  Apply Green formula \eqref{eq:green} to $E_\text{w}^\text{i}$ in $\Omega_-$, $E_\text{w}^\text{s}$ in $\Omega_+$, we have
  \begin{multline}
    \ints{\nu(y)\times E_\text{w}^\text{i}(y, d, p)\cdot H_\text{w}^\text{i}(y, -\hat{x}, q) \\- \nu(y)\times E_\text{w}^\text{i}(y, -\hat{x}, q)\cdot H_\text{w}^\text{i}(y, d, p)}(y) = 0 \label{eq:r1} 
  \end{multline}
  \begin{multline}
    \ints{\nu(y)\times E_\text{w}^\text{s}(y, d, p)\cdot H_\text{w}^\text{s}(y, -\hat{x}, q) \\- \nu(y)\times E_\text{w}^\text{s}(y, -\hat{x}, q)\cdot H_\text{w}^\text{s}(y, d, p)}(y) = 0 \label{eq:r2}
  \end{multline}
  From proposition \eqref{prp:far} we have
  \begin{multline}\label{eq:r3}
    4\pi q\cdot E_\text{w}^\infty(\hat{x}, d, p) = \ints{\nu(y)\times E_\text{w}^\text{s}(y, d, p)\cdot H_\text{w}^\text{i}(y, -\hat{x}, q) \\+ \nu(y)\times H_\text{w}^\text{s}(y, d, p)\cdot E_\text{w}^\text{i}(y, -\hat{x}, q)}(y) 
  \end{multline}
  Interchange $p, q$ and $d, \hat{x}$ respectively in \eqref{eq:r3}, we have 
  \begin{multline}\label{eq:r4}
    4\pi q\cdot E_\text{w}^\infty(\hat{x}, d, p) = \ints{\nu(y)\times E_\text{w}^\text{s}(y, -\hat{x}, q)\cdot H_\text{w}^\text{i}(y, d, p) \\+ \nu(y)\times H_\text{w}^\text{s}(y, -\hat{x}, q)\cdot E_\text{w}^\text{i}(y, d, p)}(y) 
  \end{multline}
  Subtract \eqref{eq:r3} with \eqref{eq:r4} and add \eqref{eq:r1}, \eqref{eq:r2}, together with the boundary condition 
  \begin{align*}
    \nu(y)\times E_\text{w}(y, d, p)=\nu(y)\times E_\text{w}(y, -\hat{x}, p)=0\quad\forall y\in\bdr
  \end{align*}
  the result follows. 
\end{proof}

\section{A Uniqueness Theorem}
\begin{thm}
  If $D_1$ and $D_2$ are two perfect conductors such that the electric far field patterns coincide for a fixed wave number, all incident directions and all observation directions, then $D_1=D_2$.
\end{thm}

\begin{proof}
  Let $U$ be the unbounded component of $\mathbb{R}^3\setminus(D_1\cup D_2)$. 
  By Rellich lemma, 
  $$E_{\text{w},1}^\text{s}(x, d, p)=E_{\text{w},2}^\text{s}(x, d, p)\quad\forall x\in U, d, p\in\mathbb{S}^2.$$ 
  By mixed reciprocity relation,
  $$E_{\text{w},1}^\infty(\hat{x}, z, p)=E_{\text{w},2}^\infty(\hat{x}, z, p)\quad\forall z\in U, \hat{x}, p\in\mathbb{S}^2.$$ 
  Again by Rellich lemma,
  $$E_{\text{p},1}^\text{s}(x, z, p)=E_{\text{p},2}^\text{s}(x, z, p)\quad\forall x, z\in U, p\in\mathbb{S}^2.$$

  Assume $D_1\not=D_2$, then $\exists\,\tilde{x}\in U$ such that $\tilde{x}\in\partial D_1, \tilde{x}\not\in\overline{D_2}$. Construct $\{z_n\}$ such that $z_n=\tilde{x}+\frac{1}{n}\nu(\tilde{x})\in U$ for sufficiently large $n$. From the well-posedness of the solution on $D_2$, $E_{\text{p},2}^\text{s}(\tilde{x}, \tilde{x}, p)$ is well-behaved. But $$E_{\text{p},1}^\text{s}(\tilde{x}, z_n, q)\to\infty\text{ as } z_n\to\tilde{x}\text{ and given } p\bot\nu(\tilde{x})$$ in order to fulfill the boundary condition with the incident dipole $E_{\text{p},1}^\text{i}(\tilde{x}, z_n, p)$, which becomes unbounded as $z_n\to\tilde{x}$.

\end{proof}

\section{Factorization of the Far Field Operator}

Here we set the function spaces which will be of use later.
%\begin{enumerate}
  %\item $H(\curl, \Omega) = \{v\,|\,v\in L^2(\Omega)^3, \,\curl v\in L^2(\Omega)^3\}$, with norm $$\|v\|_{H(\curl, \Omega)} = \left(\|v\|_{L^2(\Omega)^3}^2 + \|\curl v\|_{L^2(\Omega)^3}^2\right)^{\frac{1}{2}}.$$
  %\item $H(\divv, \Omega) = \{v\,|\,v\in L^2(\Omega)^3, \,\divv v\in L^2(\Omega)\}$, with norm $$\|v\|_{H(\divv, \Omega)} = \left(\|v\|_{L^2(\Omega)^3}^2 + \|\divv v\|_{L^2(\Omega)}^2\right)^{\frac{1}{2}}.$$
  %\item $H^{-\frac{1}{2}}(\divv, \bdr) = \{v\,|\,v\in H^{-\frac{1}{2}}(\bdr)^3,\, \nu\cdot v=0, \,\divv_\bdr v\in H^{-\frac{1}{2}}(\bdr)\}$.
  %\item $H^{-\frac{1}{2}}(\curl, \bdr) = \{v\,|\,v\in H^{-\frac{1}{2}}(\bdr)^3,\, \nu\cdot v=0, \,\curl_\bdr v\in H^{-\frac{1}{2}}(\bdr)\}$.
%\end{enumerate}

\begin{enumerate}
  \item $L_{2,\text{t}}^{\Div} = \{v\,|\,v\in L_2(\bdr)^3,\, \nu\cdot v=0, \,\Div v\in L_2(\bdr)\}$.
  \item $L_{2,\text{t}}^{\Curl} = \{v\,|\,v\in L_2(\bdr)^3,\, \nu\cdot v=0, \,\Curl v\in L_2(\bdr)\}$.
\end{enumerate}

\begin{prp}
  $v\to\nu\times v$ is an isomorphism from $L_{2,\text{t}}^{\Curl}$ to $L_{2,\text{t}}^{\Div}$ with inverse $w\to -\nu\times w$.
\end{prp}

\begin{dfn}
  The Maxwell problem is to find a pair of radiating solution $(E, H)\in H_\text{loc}(\curl, \mathbb{R}^3\setminus\Omega)$ to the Maxwell equations
  \begin{align*}
    \curl E - ikH &= 0\\
    \curl H + ikE &= 0
  \end{align*}
  in $\mathbb{R}^3\setminus\Omega$ with the boundary condition
  \begin{align*}
    \nu\times E = f
  \end{align*}
  where $f\in H^{-\frac{1}{2}}(\divv, \bdr)$. The data-to-pattern operator $G:H^{-\frac{1}{2}}(\divv, \bdr)\rightarrow L_\text{t}^2(\mathbb{S}^2)$ is defined by $$Gf=E^\infty$$ where $E^\infty$ denotes the far field pattern of the radiating solution $E$ of the Maxwell problem.
\end{dfn}

\begin{dfn}
  The far field operator $F:L_\text{t}^2(\mathbb{S}^2)\rightarrow L_\text{t}^2(\mathbb{S}^2)$ is defined by 
  \begin{align}
    (Fg)(\hat{x})=\int_{\mathbb{S}^2} E^\infty(\hat{x}, \theta)g(\theta)\,d\sigma(\theta),\quad\hat{x}\in \mathbb{S}^2.
  \end{align}
\end{dfn}

\begin{prp}
  \begin{enumerate}
    \item $F - F^* = \frac{ik}{8\pi}F^*F$, where $F^*$ denotes the $L^2$-adjoint of $F$.
    \item The scattering operator $S=I+\frac{ik}{8\pi^2}F$ is unitary.
    \item $F$ is normal.
  \end{enumerate}
\end{prp}

\begin{proof}
  Let $g, h\in L_{\text t}^2(\mathbb{S}^2)$ and define the Hergoltz wave functions $v^\text{i}, w^\text{i}$ with density $g, h$ respectively:
  \begin{align*}
    v^\text{i}(x) &= \int_{\mathbb{S}^2}g(\theta) e^{i k x\cdot\theta}\, d\sigma(\theta),\qquad x\in\mathbb{R}^3 \\
    w^\text{i}(x) &= \int_{\mathbb{S}^2}h(\theta) e^{i k x\cdot\theta}\, d\sigma(\theta),\qquad x\in\mathbb{R}^3 
  \end{align*}
  Let $v, w$ be solutions of the scattering problem corresponding to incident fields $v^\text{i}, w^\text{i}$, with scattered fields $v^\text{s}=v-v^\text{i}, w^\text{s}=w-w^\text{i}$ and far field patterns $v^\infty, w^\infty$ respectively. Apply Green theorem in $\Omega_R=\{x\in\mathbb{R}^3\setminus\overline{\Omega}: |x|<R\}$ with sufficiently big $R$, together with the boundary condition we have
  \begin{align}
    0 &= \int_{\Omega_R}\left(v\Delta\overline{w}-\overline{w}\Delta v\right)\,dV \\
    &= \int_{|x|=R}\left(\overline{w}\times\curl{v} - v\times\curl{\overline{w}}\right)\cdot\nu\,d\sigma.\label{int}
  \end{align} 
  Decomposing $v=v^\text{i}+v^\text{s}$ and $w=w^\text{i}+w^\text{s}$, we split \eqref{int} into the sum of the following four parts:
  \begin{align}
    &\int_{|x|=R}\left(\overline{w^\text{i}}\times\curl{v^\text{i}} - v^\text{i}\times\curl{\overline{w^\text{i}}}\right)\cdot\nu\,d\sigma,\label{int0} \\
    &\int_{|x|=R}\left(\overline{w^\text{s}}\times\curl{v^\text{s}} - v^\text{s}\times\curl{\overline{w^\text{s}}}\right)\cdot\nu\,d\sigma,\label{int1} \\
    &\int_{|x|=R}\left(\overline{w^\text{i}}\times\curl{v^\text{s}} - v^\text{s}\times\curl{\overline{w^\text{i}}}\right)\cdot\nu\,d\sigma,\label{int2} \\
    &\int_{|x|=R}\left(\overline{w^\text{s}}\times\curl{v^\text{i}} - v^\text{i}\times\curl{\overline{w^\text{s}}}\right)\cdot\nu\,d\sigma.\label{int3}
  \end{align}
  The integral \eqref{int0} vanishes by applying Green theorem in $B_R=\{x:|x|<R\}$. To evaluate the integral \eqref{int1}, we note by the radiation condition
  \begin{align}
    \overline{w^\text{s}}\times\hat{x} - \frac{1}{ik} \curl{\overline{w^\text{s}}} &= \mathcal{O}\left(\frac{1}{r^2}\right) \\
    v^\text{s}\times\hat{x} + \frac{1}{ik} \curl{v^\text{s}} &= \mathcal{O}\left(\frac{1}{r^2}\right) 
  \end{align}
  and relations between scattered fields and far field patterns
  \begin{align*}
    \overline{w^\text{s}} &= \frac{e^{-ikr}}{4\pi r}\left\{\overline{w^\infty} + \mathcal{O}\left(\frac{1}{r}\right)\right\} \\
    v^\text{s} &= \frac{e^{ikr}}{4\pi r}\left\{v^\infty + \mathcal{O}\left(\frac{1}{r}\right)\right\} 
  \end{align*}
  one obtains
  \begin{align*}
    &\left(\overline{w^\text{s}}\times\curl{v^\text{s}} - v^\text{s}\times\curl{\overline{w^\text{s}}}\right)\cdot\hat{x}\\
    &=ik\left(\overline{w^\text{s}}\times(\hat{x}\times v^\text{s})+v^\text{s}\times(\hat{x}\times\overline{w^\text{s}})\right)\cdot\hat{x} \\
    &=2ik\left(\overline{w^\text{s}}\cdot v^\text{s}-(\overline{w^\text{s}}\cdot\hat{x})(v^\text{s}\cdot\hat{x}\right))\\
    &=2ik\,\overline{w^\text{s}}\cdot v^\text{s}\\
    &=\frac{ik}{8\pi^2 r^2} \overline{w^\infty}\cdot v^\infty + \mathcal{O}\left(\frac{1}{r^3}\right)
  \end{align*}
  Hence
  \begin{align*}
    &\int_{|x|=R}\left(\overline{w^\text{s}}\times\curl{v^\text{s}} - v^\text{s}\times\curl{\overline{w^\text{s}}}\right)\cdot\nu\,d\sigma\\ 
    &\longrightarrow \frac{ik}{8\pi^2}\int_{\mathbb{S}^2}\overline{w^\infty}\cdot v^\infty\,d\sigma =\frac{ik}{8\pi^2}\left(Fg, Fh\right)_{L^2(\mathbb{S}^2)}
  \end{align*}

  To evaluate the integral \eqref{int2}, one note that it can be rearranged as
  \begin{align}
    &\int_{|x|=R}\left(\overline{w^\text{i}}\times\curl{v^\text{s}} - v^\text{s}\times\curl{\overline{w^\text{i}}}\right)\cdot\nu\,d\sigma \\
    =&-\int_{|x|=R}\left(\hat{x}\times\curl{v^\text{s}}\right)\cdot\overline{w^\text{i}}+\left(\hat{x}\times v^\text{s}\right)\cdot\curl{\overline{w^\text{i}}}\,d\sigma \label{int2a}
  \end{align}

  Substitute  
  \begin{align*}
    \overline{w^\text{i}}(x) &= \int_{\mathbb{S}^2}h(\theta) e^{-i k x\cdot\theta}\, d\sigma(\theta), \\
    \curl{\overline{w^\text{i}}(x)} &= ik\, \int_{\mathbb{S}^2}\left(h(\theta)\times\theta\right) e^{-i k x\cdot\theta}\, d\sigma(\theta)
  \end{align*}
  
  into \eqref{int2a}, the integral becomes 
  \begin{multline}\label{int2b}
    -\int_{|x|=R}\left(\hat{x}\times\curl{v^\text{s}}\right)\cdot\int_{\mathbb{S}^2}h(\theta) e^{-i k x\cdot\theta}\, d\sigma(\theta)\,d\sigma(x) \\-\int_{|x|=R}\left(\hat{x}\times v^\text{s}\right)\cdot ik\, \int_{\mathbb{S}^2}\left(h(\theta)\times\theta\right) e^{-i k x\cdot\theta}\, d\sigma(\theta) \,d\sigma(x).
  \end{multline}

  From $h(\theta)\cdot\theta=0$ and $\theta\cdot\theta=1$, by formulae
  \begin{align*}
    a\times(b\times c) &= b\,(a\cdot c) - c\,(a\cdot b) \\
    a\cdot(b\times c) &= -b\cdot(a\times c) 
  \end{align*}
  we have  
  \begin{align*}
    h(\theta)\cdot\left(\hat{x}\times\curl{v^\text{s}}\right)=&\,h(\theta)\cdot\left\{\left(\hat{x}\times\curl{v^\text{s}}\right)-\theta\left(\theta\cdot\left(\hat{x}\times\curl{v^\text{s}}\right)\right)\right\} \\
    =&\,h(\theta)\cdot\left\{\theta\times\left(\left(\hat{x}\times\curl{v^\text{s}}\right)\times\theta\right)\right\} 
  \end{align*}
  and 
  \begin{align*}
    \left(\hat{x}\times v^\text{s}\right)\cdot\left(h(\theta)\times\theta\right) = h(\theta)\cdot\left(\theta\times\left(\hat{x}\times v^\text{s}\right)\right)
  \end{align*}
  Substitute into \eqref{int2b}, the value of the integral \eqref{int2} is  
  \begin{align*}
    &-\int_{\mathbb{S}^2}\int_{|x|=R} \left\{h(\theta)\cdot\left(\hat{x}\times\curl{v^\text{s}}\right)+ik\,\left(\hat{x}\times v^\text{s}\right)\cdot\left(h(\theta)\times\theta\right)\right\} e^{-i k x\cdot\theta}\,d\sigma(x)\,d\sigma(\theta) \\
    =&-\int_{\mathbb{S}^2}h(\theta)\cdot\int_{|x|=R}\left\{\theta\times\left(\left(\hat{x}\times\curl{v^\text{s}}\right)\times\theta\right) + ik\,\theta\times\left(\hat{x}\times v^\text{s}\right)\right\} e^{-i k x\cdot\theta}\,d\sigma(x)\,d\sigma(\theta) \\
    &\longrightarrow-\left(Fg, h\right)_{L^2(\mathbb{S}^2)}.
  \end{align*}
  By the same token, the integral \eqref{int3} is $\left(g, Fh\right)_{L^2(\mathbb{S}^2)}$. Hence
  \begin{align*}
    0 = \left(g, Fh\right)_{L^2(\mathbb{S}^2)} -\left(Fg, h\right)_{L^2(\mathbb{S}^2)} + \frac{ik}{8\pi^2}\left(Fg, Fh\right)_{L^2(\mathbb{S}^2)},
  \end{align*}
  the identity
  \begin{align*}
    F - F^* = \frac{ik}{8\pi^2} F^*F
  \end{align*}
  follows.
  
  To see that $S$ is unitary, we compute
  \begin{align*}
    S^*S&=\left(I - \frac{ik}{8\pi^2}F^*\right)\left(I + \frac{ik}{8\pi^2}F\right)\\
        &= I + \frac{ik}{8\pi^2}F - \frac{ik}{8\pi^2}F^* + \frac{k^2}{64\pi^2}F^*F\\
        &= I. 
  \end{align*}
  Thus $S$ is injective as well as surjective, for $S$ is a compact perturbation of the identity. Therefore $S^*=S^{-1}$ and $SS^*=I$. Comparing $S^*S$ and $SS^*$ we can see that $F^*F=FF^*$, hence $F$ is normal.
  
\end{proof}

\begin{prp}
  $$F=-GN^*G^*.$$ 
\end{prp}

\begin{proof}
  Define auxiliary operator $\mathcal{H}:L_\text{t}^2(\mathbb{S}^2)\rightarrow H^{-\frac{1}{2}}(\divv, \bdr)$ as
  \begin{align}
    (\mathcal{H}g)(x)=\nu(x)\times\int_{\mathbb{S}^2}g(\theta)e^{ikx\cdot\theta}\,d\sigma(\theta),\quad x\in\bdr.
  \end{align}

  The adjoint operator $\mathcal{H}^*:H^{-\frac{1}{2}}(\curl, \bdr)\rightarrow L_\text{t}^2(\mathbb{S}^2)$ is 
  \begin{align}
    (\mathcal{H}^*f)(\theta)=\theta\times\left(\theta\times\ints{\left(\nu(x)\times f(x)\right)e^{-ik x\cdot\theta}}(x)\right),\quad\theta\in\mathbb{S}^2.
  \end{align}
  
  This can be verified by
  \begin{align*}
    \langle f, \mathcal{H}g\rangle &= \int_{\bdr}f(x)\cdot\overline{\left\{\nu(x)\times\int_{\mathbb{S}^2}g(\theta)\,e^{i k x\cdot\theta}\,d\sigma(\theta)\right\}}\,d\sigma(x) \\
    &=\int_{\bdr}\int_{\mathbb{S}^2} f(x)\cdot (\nu(x)\times\overline{g(\theta)})\,e^{-i k x\cdot\theta}\,d\sigma(\theta)\,d\sigma(x) \\
    &=\int_{\mathbb{S}^2}\int_{\bdr} f(x)\cdot (\nu(x)\times\overline{g(\theta)})\,e^{-i k x\cdot\theta}\,d\sigma(x)\,d\sigma(\theta) \\
    &=\int_{\mathbb{S}^2}\int_{\bdr}(f(x)\times \nu(x))\cdot\overline{g(\theta)}\,e^{-i k x\cdot\theta}\,d\sigma(x)\,d\sigma(\theta) \\
    &=\int_{\mathbb{S}^2}\int_{\bdr} \left(\theta\times\left((f(x)\times \nu(x))\times\theta\right)\right)\cdot\overline{g(\theta)}\,e^{-i k x\cdot\theta} d\sigma(x)\,d\sigma(\theta) \\
    &=\int_{\mathbb{S}^2}\left\{\theta\times\left(\int_{\bdr} (f(x)\times \nu(x))\,e^{-i k x\cdot\theta}\,d\sigma(x)\times\theta\right)\right\}\cdot\overline{g(\theta)}\,d\sigma(\theta) \\
    &=\int_{\mathbb{S}^2}\left\{\left(\theta\times\int_{\bdr}(f(x)\times \nu(x))\,e^{-i k x\cdot\theta}\,d\sigma(x)\right)\times\theta\right\}\cdot\overline{g(\theta)}\,d\sigma(\theta) \\
    &=\int_{\mathbb{S}^2}\left\{\theta\times\left(\theta\times\int_{\bdr}(\nu(x)\times f(x))\,e^{-i k x\cdot\theta}\,d\sigma(x)\right)\right\}\cdot\overline{g(\theta)}\,d\sigma(\theta) \\
    &=\langle\mathcal{H}^*f, g\rangle.
  \end{align*} 

  Given tangential $f(x)$, define $u(x)$ by
  \begin{align*}
    u(x) = \curl\curl_x\ints{\left(\nu(y)\times f(y)\right)\Phi(x,y)}(y),\quad x\in\mathbb{R}^3\setminus\bdr.
  \end{align*}
  
  From the asymptotic relation
  \begin{align*}
    \curl\curl_x\left\{a(y)\frac{e^{ik|x-y|}}{|x-y|}\right\} &= k^2\frac{e^{ik|x|}}{|x|}\left\{e^{-ik\hat{x}\cdot y}\,\hat{x}\times(\hat{x}\times a(y)) + \mathcal{O}\left(\frac{1}{|x|}\right)\right\} 
  \end{align*}
  %we have
  %\begin{align*}
  %  u(x)&=\curl\curl_x\ints{\left(\nu(y)\times f(y)\right)\Phi(x,y)}(y) \\
  %  &=\ints{\curl\curl_x\left\{\nu(y)\times f(y)\frac{e^{ik|x-y|}}{|x-y|}\right\}}(y) \\
  %  &= k^2\frac{e^{ik|x|}}{|x|}\left\{e^{-ik\hat{x}\cdot y}\,\hat{x}\times(\hat{x}\times a(y)) + O\left(\frac{|a(y)|}{|x|}\right)\right\} 
  %\end{align*} 
  the far field pattern of $u$ can be seen as $\mathcal{H}^*f$. 

  Define the electric dipole operator $N$ as %$N:H^{-\frac{1}{2}}(\divv, \bdr)\rightarrow H^{-\frac{1}{2}}(\divv, \bdr)$ as
  \begin{align}
    (Nf)(x) = \nu(x)\times\curl\curl\ints{\left(\nu(y)\times f(y)\right)\Phi(x,y)}(y),\quad x\in\bdr.
  \end{align}
  Then
  \begin{align}
    \mathcal{H}^*f = GNf.
  \end{align}
  We have
  \begin{align}
    F=-G\mathcal{H}.
  \end{align}
  hence $F=-G\mathcal{H}=-GN^*G^*$.
\end{proof}

\begin{prp}
$\Im\langle N\varphi, \varphi\rangle\geqslant 0$. 
\end{prp}

\begin{proof}
Define 
\begin{align}\label{eq:v}
  v(x) = \curl\ints{\nu(y)\times\varphi(y)\,\Phi(x,y)}(y),\qquad x\in\mathbb{R}^3\setminus\bdr.
\end{align}
Note that 
  \begin{align*}
  v_\pm(x) &= \pv\ints{\nabla_x\Phi(x, y)\times\left(\nu(y)\times\varphi(y)\right)}(y)\mp\frac{1}{2}\nu(x)\times(\nu(x)\times\varphi(x))\\ 
  &= \pv\ints{\nabla_x\Phi(x, y)\times\left(\nu(y)\times\varphi(y)\right)}(y)\pm\frac{1}{2}\varphi(x) 
  \end{align*}
and $\divv v=0, \Delta v + k^2 v=0$. 

set $a=\overline{v}, b=v$ in vector Green formula
\begin{align*}
  \intvv{a\cdot\Delta b+\curl a\cdot\curl b + \divv a\divv b}=\intss{-(\nu\times\curl b)\cdot a+(\nu\cdot a)\divv b}
\end{align*}
we can see that
  \begin{align*}
    \langle N\varphi, \varphi\rangle &=\left\langle \nu\times\curl v, v_+-v_-\right\rangle \\
    &=\ints{\nu\times\curl v\cdot(\overline{v_+} - \overline{v_-})} \\
    &=\ints{\nu\times\curl v\cdot\overline{v_+}}- \ints{\nu\times\curl v\cdot\overline{v_-}}\\
    &=-\int_{\Omega\cup B_R}k^2|v|^2 - |\curl v|^2\,dV + \int_{|x|=R}\hat{x}\times\curl v\cdot\overline{v}\,d\sigma \\
    &=-\int_{\Omega\cup B_R}k^2|v|^2 - |\curl v|^2\,dV + ik\,\int_{|x|=R}|v|^2\,d\sigma+\mathcal{O}\left(\frac{1}{R}\right)
  \end{align*}
Take the imaginary part and let $R\to\infty$, we have
  \begin{align*}
    \Im\langle N\varphi,\varphi\rangle = k\lim_{R\to\infty}\int_{|x|=R}|v|^2\,d\sigma = \frac{k}{16\pi^2}\int_{\mathbb{S}^2}|v^\infty|^2\,d\sigma\geqslant 0.
  \end{align*}
\end{proof}

\begin{prp}
  \label{pr:lip}
  Given a bounded Lipschitz domain $\Omega$, the followings hold:
  \begin{enumerate}
    \item\label{it:lip1}
      There exists a regular family of cones $\{\zeta\}$.
    \item There exists a sequence of $C^{\infty}$ domains 
      $\Omega_i\subset \Omega$ and corresponding homeomorphisms 
      $\Lambda_j:\bdr\to\bdr_i$ such that 
      $\sup_{x\in\bdr}|\Lambda_j(x)-x|\to 0$ as $j\to\infty$ and
      for all $j$ and all $x\in\bdr$, $\Lambda_j(x)\in\zeta(x)$.
    \item There exist positive functions 
      $\omega_j:\bdr\to\mathbb{R}^+$ bounded away from zero and infinity 
      uniformly in $j$ such that 
      \begin{enumerate}
	\item For any measurable set $V\subset\bdr$
	  \begin{align*}
	    \int_V \omega_j\,d\sigma=\int_{\Lambda_j(V)}\,d\sigma_j.
	  \end{align*}
	\item $\omega_j(x)\to 1$ pointwise a.e. for $x\in\bdr$. 
      \end{enumerate}
    \item $\nu(\Lambda_j(x))\to\nu(x)$ pointwise a.e. for $x\in\bdr$. 
    \item\label{it:kk}
      There exists a real-valued $C^{\infty}$ vector field $h$ such that for
      all $j$ and $x\in\bdr$, 
      $\nu(\Lambda_j(x))\cdot h(\Lambda_j(x))\geqslant \kappa>0$, where 
      $\kappa$ depends on the Lipschitz character of $\Omega$. 
      Without loss of generality, $\kappa<1$.
  \end{enumerate}
\end{prp}

\begin{lmm}[Rellich identity]
  For a complex-valued $C^{\infty}(\overline{\Omega})$ vector field $E$ and
  a real-valued $C^{\infty}(\mathbb{R}^3)$ vector field $h$
  \begin{multline}
    \ints{\Bigl\{\frac{1}{2}|E|^2 (h\cdot\nu)-\re\bigl((\overline{E}\cdot h)
    (E\cdot\nu)\bigr)\Bigr\}}\\
    =\intv{\re\Bigl\{\frac{1}{2}|E|^2\divv h-(\overline{E}\cdot h)\divv E
    -\overline{E}\cdot(\nabla h)E+(h\times\overline{E})\cdot\curl E\Bigr\}},
    \label{eq:Rellich}
  \end{multline}
  where $\overline{E}\cdot(\nabla h)E$ denotes the quadratic form
  $\Sigma_{i,j}(D_i h_j) E_i \overline{E_j}$.
\end{lmm}
\begin{proof}
  It is evident from
  \begin{multline*}
    \divv\Bigl\{\frac{1}{2}|E|^2 h-\re\bigl((\overline{E}\cdot h)E\bigr)
    \Bigr\}\\
    =\re\Bigl\{\frac{1}{2}|E|^2\divv h-(\overline{E}\cdot h)\divv E-
    \overline{E}\cdot(\nabla h)E+(h\times\overline{E})\cdot\curl E\Bigr\}
  \end{multline*}
  and Divergence theorem. 
\end{proof}

\begin{lmm}\label{lm:start}
  For a complex-valued $C^{\infty}(\overline{\Omega})$ vector field $E$
  \begin{align}
    \ints{|E|^2} &\lesssim \ints{|E_{\text{n}}|^2}+\intv{|E|^2+|\curl E|^2+|\divv E|^2}
    \label{eq:En}\\
    \ints{|E|^2} &\lesssim \ints{|E_{\text{t}}|^2}+\intv{|E|^2+|\curl E|^2+|\divv E|^2}
    \label{eq:Et}.
  \end{align}
  If $E\in C^{\infty}(\overline{\Omega_+})$ and decays at infinity then the
  above hold with $\Omega$ replaced by $\Omega_+$.
\end{lmm}
\begin{proof}
  Let $h$ be the real-valued vector field which satisfies proposition 
  \ref{pr:lip}, item \eqref{it:kk}, i.e. $h\cdot\nu\geqslant\kappa
  >0$ on $\bdr$. Decomposing $E$, $h$ into mutually orthogonal parts 
  $E=E_{\text{t}}+E_{\text{n}}$, $h=h_{\text{t}}+h_{\text{n}}$, we have 
  \begin{multline*}
    \frac{1}{2}|E|^2(h\cdot\nu)-\re\bigl((\overline{E}\cdot h)(E\cdot\nu)
    \bigr)\\
    =\frac{1}{2}|E_{\text{t}}|^2(h\cdot\nu)-\frac{1}{2}|E_{\text{n}}|^2(h\cdot\nu)-
    \re\bigl((\overline{E_{\text{t}}}\cdot h_{\text{t}})(E_{\text{n}}\cdot\nu)\bigr),
  \end{multline*}
  thus the Rellich identity \eqref{eq:Rellich} is rewritten as
  \begin{align}
    \ints{\frac{1}{2}|E_{\text{t}}|^2(h\cdot\nu)}&=
    \ints{\frac{1}{2}|E_{\text{n}}|^2(h\cdot\nu)}+\Theta_1+\Theta_2,
    \label{eq:Rellich1}
  \end{align}
  where
  \begin{align*}
    \Theta_1 &:=\ints{\re\bigl((\overline{E_{\text{t}}}\cdot h_{\text{t}})(E_{\text{n}}\cdot\nu)\bigr)},
    \\
    \Theta_2 &:=\intv{\re\Bigl\{\frac{1}{2}|E|^2\divv h-(\overline{E}\cdot h)
    \divv E-\overline{E}\cdot(\nabla h)E+(h\times\overline{E})\cdot\curl E
    \Bigr\}}
  \end{align*}
  In view of \eqref{eq:Rellich1} and $h\cdot\nu\geqslant\kappa>0$
  we have
  \begin{align}\label{eq:ineq1}
    \frac{1}{2}\kappa\ints{|E_{\text{t}}|^2}&\leqslant\frac{1}{2}\ints{|E_{\text{n}}|^2}
    +\Theta_1+\Theta_2.
  \end{align} 
  By Young's inequality 
  \begin{align*}
    ab\leqslant\varepsilon a^2+\frac{1}{\varepsilon}b^2\quad\forall
    \varepsilon>0
  \end{align*}
  \eqref{eq:ineq1} becomes
  \begin{align}
    \ints{|E|^2}\lesssim\ints{|E_{\text{n}}|^2}+\intv{|E|^2+|E||\curl E|+|E||\divv E|}
    \label{eq:Ennn}
  \end{align}
  Similarly, from \eqref{eq:Rellich1} and (\ref{it:kk}) $h\cdot\nu\geqslant\kappa>0$  we have
  \begin{equation}\label{eq:ineq2}
    \begin{split}
      \frac{1}{2}\kappa\ints{|E_{\text{n}}|^2} 
      &\leqslant \frac{1}{2}\ints{|E_{\text{t}}|^2}-\Theta_1-\Theta_2\\
      &\leqslant\frac{1}{2}\ints{|E_{\text{t}}|^2}+|\Theta_1|+|\Theta_2|,
    \end{split}
  \end{equation}
  hence by Young's inequality \eqref{eq:ineq2} becomes
  \begin{align}
    \ints{|E|^2}\lesssim\ints{|E_{\text{t}}|^2}+\intv{|E|^2+|E||\curl E|+|E||\divv E|}
    \label{eq:Ettt}.
  \end{align}
  Once by Young's inequality 
  \begin{align*}
    \intv{|E|^2+|E||\curl E|+|E||\divv E|}\lesssim\intv{|E|^2+|\curl E|^2+
    |\divv E|^2},
  \end{align*}
  and we may rewrite \eqref{eq:Ennn}, \eqref{eq:Ettt} into \eqref{eq:En}, 
  \eqref{eq:Et} respectively.
\end{proof}

\begin{lmm}\label{lm:tt}
  For the complex-valued $C^{\infty}(\overline{\Omega})$ vector field $E$
  which satisfies $(\triangle+k^2)E=0$ and $\divv E=0$ in $\Omega$,  
  \begin{align*}
    \nltwo{E} + \nltwo{\curl E} &\approx \nltwod{\nu\times\curl E}
  \end{align*}
\end{lmm}
\begin{proof}
  Setting $a=\overline{E}$ and $b=E$ in vector Green's theorem 
  \begin{align*}
    \intvv{a\triangle b+\curl a\cdot\curl b+\divv a\cdot\divv b}=\intss{(\nu\times a)\cdot\curl b+(\nu\cdot a)\divv b}
  \end{align*}
  we have
  \begin{align*}
    \ints{(\nu\times\overline{E})\cdot\curl{E}+(\overline{E}\cdot\nu)\divv E}
    =\intv{|\curl E|^2+|\divv E|^2-k^2|E|^2}.
  \end{align*}
  
  
  The above identity becomes
  \begin{multline*}
    \intv{|E|^2+|\curl E|^2+|\divv E|^2}\\\lesssim\Bigl|\ints{(\nu\times
    \overline{E})\cdot\curl E}\Bigr|+\ints{|E\cdot\nu||\divv E|}.
  \end{multline*}
  Once by $|E\cdot\nu|\leqslant |E|$ and Young's inequality
  \begin{align*}
    \ints{|E\cdot\nu||\divv E|}\leqslant\text{(small)}\ints{|E|^2}+
    \text{(large)}\ints{|\divv E|^2},
  \end{align*}
  which turns \eqref{eq:En} into
  \begin{align}
    \ints{|\nu\times E|^2}\lesssim\ints{|E\cdot\nu|^2+|\divv E|^2}+\Bigl|
    \ints{(\nu\times\overline{E})\cdot\curl{E}}\Bigr|.
    \label{eq:ee} 
  \end{align}
  % We begin with a simple observation: if $a$, $b$, $c$, $d$ are 
  % non-negative and $d^2\lesssim a^2+b^2+c^2$, then $d\lesssim a+b+c$.
  % Indeed, if there exists $\alpha>0$ such that 
  % $d^2\leqslant\alpha(a^2+b^2+c^2)$, then 
  % $d^2\leqslant\alpha(a^2+b^2+c^2+2ab+2bc+2ac)$ 
  % because $a,b,c$ and $\alpha$ are nonnegative; hence 
  % $d\leqslant\sqrt{\alpha}(a+b+c)$.  
  Together with the result of lemma \ref{lm:start}, we have
  \begin{equation}
    \begin{split}\label{eq:nl1}
      \nltwo{E} &\lesssim \nltwo{E_{\text{n}}}+\nltwo{(\curl E)_{\text{t}}}+\nltwo{\divv E},\\
      \nltwo{E} &\lesssim \nltwo{E_{\text{t}}}+\nltwo{(\curl E)_{\text{t}}}+\nltwo{\divv E}.
    \end{split}
  \end{equation}
  By writing 
  $H=\frac{1}{ik}\curl E$, \eqref{eq:nl1} becomes
  \begin{align}
    \nltwo{E} &\lesssim \nltwo{E_{\text{n}}}+\nltwo{H_{\text{t}}},\label{eq:Enn}\\
    \nltwo{E} &\lesssim \nltwo{E_{\text{t}}}+\nltwo{H_{\text{t}}}.\label{eq:Ett}
  \end{align}
  From $\curl\curl E=-\triangle E+\nabla\divv E$ we are free to permute $E$ 
  and $H$ in \eqref{eq:Enn}, \eqref{eq:Ett} and obtain
  \begin{align}
    \nltwo{H} &\lesssim \nltwo{H_{\text{n}}}+\nltwo{E_{\text{t}}},\label{eq:int}\\
    \nltwo{H} &\lesssim \nltwo{H_{\text{t}}}+\nltwo{E_{\text{t}}}.\label{eq:intt}
  \end{align}
  By \eqref{eq:Ett} and \eqref{eq:int},
  \begin{equation}
    \begin{split}\label{eq:e1}
      \nltwo{E} &\lesssim \nltwo{E_{\text{t}}}+\nltwo{H_{\text{t}}}\\
      &\lesssim \nltwo{E_{\text{t}}}+\nltwo{H_{\text{t}}}+\nltwo{H_{\text{n}}}\\
      &\lesssim \nltwo{E_{\text{t}}}+\nltwo{H}\\
      &\lesssim\nltwo{E_{\text{t}}}+\nltwo{H_{\text{n}}}+\nltwo{E_{\text{t}}}\\
      &\lesssim\nltwo{H_{\text{n}}}+\nltwo{E_{\text{t}}}.
    \end{split}
  \end{equation}
  From \eqref{eq:e1}, \eqref{eq:int} and $\nltwo{E_{\text{t}}}+\nltwo{H_{\text{n}}}\lesssim\nltwo{E}
  +\nltwo{H}$, we have 
  \begin{align}
    \nltwo{E}+\nltwo{H} &\approx \nltwo{E_{\text{t}}}+\nltwo{H_{\text{n}}}.\label{eq:fin}
  \end{align}
  Once by permutting $E$ and $H$ in \eqref{eq:fin} we have
  \begin{align}
    \nltwo{H}+\nltwo{E} &\approx \nltwo{H_{\text{t}}}+\nltwo{E_{\text{n}}},\label{eq:finn}
  \end{align}
  By $\nltwod{\cdot}\equiv\nltwo{\cdot}+\nltwo{\Div(\cdot)}$ and $\Div(\nu\times E)=-\nu\cdot\curl E$, \eqref{eq:finn} is written as
  \begin{align}
    \nltwo{E} + \nltwo{\curl E} &\approx \nltwod{\nu\times\curl E}
  \end{align}
  as claimed.
\end{proof}

\begin{prp}
  $-\langle N_i\varphi,\varphi\rangle \geqslant c \|\varphi\|^2$.
\end{prp}

\begin{proof}
  \begin{align*}
   -\langle N_i\varphi,\varphi\rangle =\int_{\Omega\cup B_R}|v|^2 + |\curl v|^2\,dV +\,\int_{|x|=R}|v|^2\,d\sigma+\mathcal{O}\left(\frac{1}{R}\right)
  \end{align*}
  As $R\to\infty$, 
  \begin{align*}
    -\langle N_i\varphi,\varphi\rangle =\int_{\mathbb{R}^3}|v|^2 + |\curl v|^2\,dV\geqslant \int_{\bdr}|v|^2 + |\curl v|^2\,d\sigma.  
  \end{align*}
  Recall that 
  \begin{align*}
  v = \curl\ints{\nu(y)\times\varphi(y)\,\Phi(x,y)}(y)
  \end{align*}
  Set $E=v$ in lemma \ref{lm:tt}, we have
  \begin{align*}
    \nltwo{v} + \nltwo{\curl v} &\approx \nltwod{\nu\times\curl v}
  \end{align*}   
  Hence
  \begin{align*}
    -\langle N_i\varphi,\varphi\rangle \geqslant c\,\nltwod{\nu\times\curl v}^2 = c\,\nltwod{N_i\varphi}^2 \geqslant c\,\nltwoc{\varphi}^2.
  \end{align*}
\end{proof}

\begin{prp}
  For $z\in\mathbb{R}^3$ and a fixed $d\in\mathbb{S}^2$, define $\varphi_z\in L^2(\mathbb{S}^2)$ by
  \begin{align*}
    \varphi_z(\hat{x}) = ik\,(\hat{x}\times d) e^{ik\hat{x}\cdot z}\qquad\hat{x}\in\mathbb{S}^2,
  \end{align*}
  then $\varphi_z$ belongs to the range of $G$ if and only if $z\in\Omega$.
\end{prp}

\begin{proof}
  Assume first $z\in\Omega$. For $x\in\mathbb{R}^3\setminus\Omega$ define
  \begin{align*}
    v(x)=\curl_x d\,\Phi(x, z)=\curl_x d\,\frac{e^{ik|x-z|}}{4\pi|x-z|}
  \end{align*}
  and $f=v|_\bdr$. The far field pattern of $v$, denoted by $v^\infty$, is 
  \begin{align*}
    v^\infty(\hat{x}) = ik\,(\hat{x}\times d)\,e^{ik\hat{x}\cdot z},\qquad\hat{x}\in\mathbb{S}^2,
  \end{align*}
  which is identical to $\varphi_z$. From $Gf=v^\infty=\varphi_z$, $\varphi_z$ belongs to the range of $G$.  
  
  Now assume $z\not\in\Omega$ and there exists $f$ with $Gf=\varphi_z$. Let $v$ be the radiating solution of the Maxwell problem with boundary data $f$ and $v^\infty=Gf$ be the far field pattern of $v$. Note that the far field pattern of $\curl d\,\Phi(\cdot, z)$ is $\varphi_z$, from Rellich lemma $v(x)=\curl d\,\Phi(x, z)$ for all $x$ outside of any sphere which contains both $z$ and $\Omega$. By analytic continuation, $v$ and $\curl d\,\Phi(\cdot, z)$ coincide on $\mathbb{R}^3\setminus(\overline{\Omega}\cup\{z\})$. But if $z\not\in\overline{\Omega}$, then $\curl d\,\Phi(x, z)$ is singular on $x=z$, while $v$ is analytic on $\mathbb{R}^3\setminus\overline{\Omega}$, a contradiction. Otherwise if $z\in\bdr$, then $x\mapsto\curl d\,\Phi(x, z)$ for $x\in\bdr, x\neq z$, is in $H^{\frac{1}{2}}(\bdr)$. But $\curl d\,\Phi(x, z)$ does not belong to $H_\text{loc}(\curl, \mathbb{R}^3\setminus\Omega)$ or $H(\curl, \Omega)$, for $\curl\Phi(x, z)=\mathcal{O}\left(1/|x-z|^2\right)$ if $x\to z$.

\end{proof}

\section{An Illustration Using Spherical Wave Expansion}

In this section we follow the notations and treatments in \cite{jackson} closely.

The spherical Bessel and Hankel functions which denoted by $j_l(x)$, $n_l(x)$, $h_l(x)$, $h_l^{(1)}(x)$ are defined as

\begin{align}
  j_l(x) &= \sqrt{\frac{\pi}{2x}}J_{l+\frac{1}{2}}(x) \\
  n_l(x) &= \sqrt{\frac{\pi}{2x}}N_{l+\frac{1}{2}}(x) \\
  h_l(x) &= \sqrt{\frac{\pi}{2x}}\left(J_{l+\frac{1}{2}}(x) + iN_{l+\frac{1}{2}}(x)\right) \\
  h_l^{(1)}(x) &= \sqrt{\frac{\pi}{2x}}\left(J_{l+\frac{1}{2}}(x) - iN_{l+\frac{1}{2}}(x)\right)
\end{align}

The spherical Bessel functions satisfy the recursion formulae
\begin{align}
  f_l(x)  &= \frac{x}{2l+1}\left(f_{l-1}(x) + f_{l+1}(x)\right) \\
  f_l'(x) &= \frac{1}{2l+1}\left(lf_{l-1}(x) - (l+1)f_{l+1}(x)\right)
\end{align}
where $f_l(x)$ is any one of the function $j_l(x)$, $n_l(x)$, $h_l(x)$, $h_l^{(1)}(x)$.

The orbital angular momentum operator $\bf{L}$ is defined by
\begin{align}
  {\bf L} = \frac{1}{i}\,x\times\nabla
\end{align}
where $x$ is the position vector. 

Define the operators $L_x,L_y,L_z$ to be the cartesian components of the orbital angular-momentum operator $\bf{L}$ respectively, and let $L^2=L_x^2+L_y^2+L_z^2$. 

\begin{align}
  -\left\{\frac{1}{\sin\vartheta}\frac{\partial}{\partial\vartheta}\left(\sin\vartheta\frac{\partial}{\partial\vartheta}\right)+\frac{1}{\sin^2 \vartheta}\frac{\partial^2}{\partial\varphi^2}\right\} Y_l^m = l(l+1) Y_l^m
\end{align}

\begin{align}
  L_+ &= L_x+iL_y = e^{i\varphi}\left(\frac{\partial}{\partial\vartheta} + i\cot\vartheta\frac{\partial}{\partial\varphi}\right) \\
  L_- &= L_x-iL_y = e^{-i\varphi}\left(-\frac{\partial}{\partial\vartheta} + i\cot\vartheta\frac{\partial}{\partial\varphi}\right) \\
  L_z &= -i\frac{\partial}{\partial\varphi}
\end{align}
  
The vector spherical harmonic $X_l^m(\vartheta,\varphi)$ is defined by 
\begin{align}
  X_l^m(\vartheta, \varphi) = \frac{1}{\sqrt{l(l+1)}}\,{\bf L}\,Y_l^m(\vartheta, \varphi)
\end{align}

With $\hat{x} = \frac{x}{\|x\|}$, we have the orthogonal relations

\begin{align}\label{orth1}
  &\int\overline{X_l^m}\cdot X_{l'}^{m'}\,d\Omega =\delta_{ll'}\delta_{mm'}\\
  &\int\overline{X_l^m}\cdot (\hat{x}\times X_{l'}^{m'})\,d\Omega =0
\end{align}

\begin{align}\label{orth2}
  \hat{x}\cdot X_l^m(\vartheta, \varphi) = 0,
\end{align}

\begin{align}
  L_+Y_l^m &= \sqrt{(l-m)(l+m+1)}Y_l^{m+1}\\
  L_-Y_l^m &= \sqrt{(l+m)(l-m+1)}Y_l^{m-1}\\
  L_zY_l^m &= mY_l^m
\end{align}

\begin{multline}
  \nabla\times f_l(r)\,X_l^m(\vartheta, \varphi)\\=i\hat{x}\,\sqrt{l(l+1)}\,\frac{f_l(r)}{r}\,Y_l^m(\vartheta, \varphi) +\frac{1}{r}\,\frac{\partial}{\partial r}(r f_l(r))\,\hat{x}\times X_l^m(\vartheta, \varphi) 
\end{multline}
where $f_l(x)$ is any one of the function $j_l(x)$, $n_l(x)$, $h_l(x)$, $h_l^{(1)}(x)$.

\begin{align}
  g_l(kr) = A^{(1)}_l h^{(1)}_l(kr) + A^{(2)}_l h^{(2)}_l(kr)
\end{align}

% for future use
\begin{align}
  &\int\overline{f_l(r)X_l^m}\cdot g_l(r)X_{l'}^{m'}\,d\Omega =\overline{f_l}g_l\delta_{ll'}\delta_{mm'}\\
  &\int\overline{f_l(r)X_l^m}\cdot (\nabla\times g_l(r)X_{l'}^{m'})\,d\Omega =0
\end{align}

\begin{multline}
  \int\overline{\nabla\times f_l(r)X_l^m}\cdot (\nabla\times g_l(r)X_{l'}^{m'})\,d\Omega\\=k^2\delta_{ll'}\delta_{mm'}\left(\overline{f_l}g_l + \frac{1}{k^2 r^2}\frac{\partial}{\partial r}(r\overline{f_l}\frac{\partial}{\partial r}(rg_l))\right), 
\end{multline}
 
where $f_l$, $g_l$ are any of the spherical bessel functions.

The addition theorem for spherical harmonics

\begin{align}
  P_l(\cos\gamma) = \frac{4\pi}{2l+1}\sum_{m=-l}^l \overline{Y_l^m(\vartheta',\varphi')}Y_l^m(\vartheta, \varphi)
\end{align}
where $\cos\gamma=\cos\vartheta\cos\vartheta' + \sin\vartheta\sin\vartheta'\cos(\varphi-\varphi')$

The multipole expansion of the plane wave is
\begin{align}
  E_\text{w}(x) = \sum_{l=1}^\infty i^l\sqrt{4\pi(2l+1)}\left(j_l(kr)X_l^{\pm 1}\pm
                  \frac{1}{k}\nabla\times j_l(kr)X_l^{\pm 1}\right)
\end{align}

This is shown as follows. First note the Jacobi-Anger expansion
\begin{align}
  e^{i\bf{k}\cdot\bf{x}} &= \sum_{l=0}^{\infty}i^l(2l+1)j_l(kr)P_l(\cos\gamma) \\
              &= \sum_{l=0}^{\infty}i^l\sqrt{4\pi(2l+1)}j_l(kr)Y_l^0(\cos\gamma)
\end{align}
where $\gamma$ is the angle between $\bf{k}$ and $\bf{x}$.

We consider an equivalent expansion for a circularly polarized plane wave with helicity $\pm$ along the $z$ axis:
\begin{align}
  E(x) &= (\varepsilon_1\pm i\varepsilon_2)e^{ikz} \\
  B(x) &= \varepsilon_3\times E = \mp iE
\end{align}

\begin{align}
  E(x)&=\sum_{l,m}\left\{a_\pm(l, m)j_l(kr)X_l^m + \frac{i}{k} b_\pm(l, m)\nabla\times j_l(kr)X_l^m\right\} \\
  B(x)&=\sum_{l,m}\left\{\frac{-i}{k}a_\pm(l, m)j_l(kr)X_l^m + b_\pm(l, m)\nabla\times j_l(kr)X_l^m\right\}
\end{align}

From the orthogonality of $X_l^m$, we have 
\begin{align}
  a_\pm(l, m)j_l(kr) &= \int \overline{X_l^m}\cdot E(x)\,d\Omega \\
  b_\pm(l, m)j_l(kr) &= \int \overline{X_l^m}\cdot B(x)\,d\Omega
\end{align}

In view of the expression of $E, B$ and the definition of $X_l^m$, after some manipulation we observed 
\begin{align}
  a_\pm(l, m)j_l(kr) = \frac{1}{\sqrt{l(l+1)}}\int\overline{L_{\mp}Y_l^m}e^{ikz}\,d\Omega
\end{align}

\begin{align}
  a_\pm(l, m)j_l(kr) = \frac{\sqrt{(l\pm m)(l\mp m + 1)}}{\sqrt{l(l+1)}}\int \overline{Y_l^{m\pm 1}}e^{ikz}\,d\Omega
\end{align}

Insert the Jacobi-Anger expansion for $e^{ikz}$, the orthogonality of $Y_l^m$ leads to
\begin{align}
  a_\pm(l, m) = i^l\sqrt{4\pi(2l+1)}\delta_{m,\pm 1}
\end{align}

From $B=\mp iE$, we obtain immediately
\begin{align}
  b_\pm(l, m) = \mp i a_\pm(l, m)
\end{align}

The scattered electric field is
\begin{multline}
  E_\text{s}(x) = \frac{1}{2}\,\sum_{l=1}^\infty i^l\sqrt{4\pi(2l+1)}\cdot\\ \left(\frac{j_l(k)}{h_l(k)}\, h_l(kr)X_l^{\pm 1}\pm \frac{1}{k}\frac{kj_l'(k)+j_l(k)}{kh_l'(k)+h_l(k)}\,\nabla\times h_l(kr)\,X_l^{\pm 1}\right)
\end{multline}

The far field pattern of the scattered electric field is
\begin{multline}
  E_\infty(\hat{x}) = \frac{-i}{2k}\,\sum_{l=1}^\infty\sqrt{4\pi(2l+1)}\cdot\\ \left(\frac{j_l(k)}{h_l(k)}\,\hat{x}\times X_l^{\pm 1}\mp \frac{kj_l'(k)+j_l(k)}{kh_l'(k)+h_l(k)}\,X_l^{\pm 1}\right)
\end{multline}

Hence $\{X_l^{\pm 1}, \hat{x}\times X_l^{\pm 1}\}$ are the eigenfunctions of the far field operator with corresponding eigenvalues $\{\frac{\pm i\sqrt{\pi(2l+1)}}{k}\frac{kj_l'(k)+j_l(k)}{kh_l'(k)+h_l(k)}, \frac{-i\sqrt{\pi(2l+1)}}{k}\frac{j_l(k)}{h_l(k)}\}$.

We wish to compute
\begin{align}\label{sum}
  \sum_m\frac{|\langle\hat{x}\times E_w, \phi_m\rangle|^2}{|\lambda_m|}
\end{align}
where the index $m$ runs through the eigenpairs $\{\phi_m, \lambda_m\}$ of the far field operator and $\langle\cdot,\cdot\rangle$ denote the $L^2(\mathbb{S}^2)$ inner product. Note that
\begin{align}
  \hat{x}\times E_\text{w}(x) = \sum_{l=1}^\infty i^l\sqrt{4\pi(2l+1)}\left(j_l(kr)\hat{x}\times X_l^{\pm 1}\mp\frac{1}{k r}\frac{\partial}{\partial r}(r j_l(kr))X_l^{\pm 1}\right)
\end{align}
In view of the vector formula $$(a\times b)\cdot(c\times d)=(a\cdot c)(b\cdot d)-(a\cdot d)(b\cdot c)$$ we have
\begin{align}
  (\hat{x}\times X_l^{\pm 1})\cdot (\hat{x}\times\overline{X_{l'}^{\pm 1}}) &= (\hat{x}\cdot\hat{x})(X_l^{\pm 1}\cdot\overline{X_{l'}^{\pm 1}}) - (\hat{x}\cdot\overline{X_{l'}^{\pm 1}})(X_l^{\pm 1}\cdot\hat{x}) \\
  &= X_l^{\pm 1}\cdot\overline{X_{l'}^{\pm 1}}
\end{align}
Together with orthogonal relations \eqref{orth1} and \eqref{orth2}, the infinite sum \eqref{sum} becomes
\begin{align}
  \frac{4\sqrt\pi}{k}\sum_l \sqrt{2l+1}\left(\frac{|j_l(kr)|^2}{\left|\frac{j_l(k)}{h_l(k)}\right|} + \frac{|\frac{1}{k r}\frac{\partial}{\partial r}(r j_l(kr))|^2}{\left|\frac{kj_l'(k)+j_l(k)}{kh_l'(k)+h_l(k)}\right|}\right)
\end{align}
We wish to investigate the convergence of this sum.

Using the asymptotic relations of $j_l(k), h_l(k)$
\begin{align}
  j_l(k) &= \frac{k^l}{1\cdot 3\cdots(2l+1)}\left(1 + \mathcal{O}\left(\frac{1}{l}\right)\right) \\
  h_l(k) &= \frac{1\cdot 3\cdots(2l-1)}{ik^{l+1}}\left(1 + \mathcal{O}\left(\frac{1}{l}\right)\right) 
\end{align}
we have
\begin{align}
  \frac{j_l(k)}{h_l(k)} &= -i\frac{k^{2l+1}}{(2l-1)!!(2l+1)!!}\left(1 + \mathcal{O}\left(\frac{1}{l}\right)\right) \\
  \frac{kj_l'(k)+j_l(k)}{kh_l'(k)+h_l(k)} &= ?\left(1 + \mathcal{O}\left(\frac{1}{l}\right)\right)
\end{align}

\nocite{*}
\bibliographystyle{ieeetr}
\bibliography{draft}
\end{document}

%%%%%%%%%%%%%%%%%%%%%%%%%%%%%%%%%%%%%%%%%%%%%%%%%%%%%%%%%%%%%%%%%%%%%%%%%%%%%%%

For $x, z\in\Omega_+$, $\hat{x}, d\in\mathbb{S}^2$, let $u_\text{w}^\text{s}(x, d)$ be the scattered field due to the incident plane wave $u_\text{w}^\text{i}(x, d) = e^{ikx\cdot d}$ and $u_\text{p}^\infty(\hat{x}, z)$ be the far field pattern of the scattered field $u_\text{p}^\text{s}(x, z)$ due to the incident point source $u_\text{p}^\text{i}(x, z) = \Phi(x, z)$.

\begin{thm}[Mixed Reciprocity Relation]
  $$u_\text{w}^\text{s}(z, -\hat{x}) = 4\pi u_\text{p}^\infty(\hat{x}, z)$$
\end{thm}

\begin{proof}
  \begin{align}
    u_\text{w}^\text{s}(z, -\hat{x}) = \ints{\left\{u_\text{w}^\text{s}(y, -\hat{x})\frac{\partial\Phi(z, y)}{\partial\nu(y)} - \frac{\partial u_\text{w}^\text{s}(y, -\hat{x})}{\partial\nu(y)}\Phi(z, y)\right\}}(y)\label{eq:q1}
  \end{align}
  \begin{align}
    u_\text{p}^\infty(\hat{x}, z) &= \frac{1}{4\pi}\ints{\left\{u_\text{p}^\text{s}(y, z)\frac{\partial e^{-ik\hat{x}\cdot y}}{\partial\nu(y)} - \frac{\partial u_\text{p}^\text{s}(y, z)}{\partial\nu(y)}e^{-ik\hat{x}\cdot y}\right\}}(y) \\ &=\frac{1}{4\pi}\ints{\left\{u_\text{p}^\text{s}(y, z)\frac{\partial u_\text{w}^\text{i}(y, -\hat{x})}{\partial\nu(y)} - \frac{\partial u_\text{p}^\text{s}(y, z)}{\partial\nu(y)}u_\text{w}^\text{i}(y, -\hat{x})\right\}}(y)\label{eq:q2}
  \end{align}
  \begin{align}
    0 = \ints{\left\{u_\text{p}^\text{s}(y, z)\frac{\partial u_\text{w}^\text{s}(y, -\hat{x})}{\partial\nu(y)} - \frac{\partial u_\text{p}^\text{s}(y, z)}{\partial\nu(y)}u_\text{w}^\text{s}(y, -\hat{x})\right\}}(y)\label{eq:q3}
  \end{align}

  Mulitiply \eqref{eq:q3} with $\frac{1}{4\pi}$ and add to \eqref{eq:q2}, we have 
  \begin{align}
    u_\text{p}^\infty(\hat{x}, z) = \frac{1}{4\pi}\ints{u_\text{p}^\text{s}(y, z)\frac{\partial u(y, -\hat{x})}{\partial\nu(y)}}(y).\label{eq:q4}
  \end{align}

\end{proof}

\section{A Result that May Be of Interest}

\begin{prp}[\cite{douglas}, Theorem 1]
  Let $X, Y, Z$ be Hilbert spaces and there are bounded operators $A:X\rightarrow Z$, $B:Y\rightarrow Z$. Then the followings are equivalent:
  \begin{enumerate}
    \item $\range(A)\subseteq\range(B)$.
    \item $AA^*\leqslant\lambda\,BB^*$ for some $\lambda\geqslant 0$.
    \item There exists a $C: X\rightarrow Y$ such that $A=BC$.
  \end{enumerate}
  Moreover, if all above conditions hold, then there exists an unique $C: X\rightarrow Y$ such that
  \begin{itemize}
    \item $\|C\|^2 = \inf\{\lambda\,|\,AA^*\leqslant\lambda\,BB^*\}$,
    \item $\nul(A)\subseteq\nul(C)$,
    \item $\range(C)\subseteq\overline{\range(B^*)}$.
  \end{itemize}
\end{prp}

\begin{lmm}\label{lmm:0} 
  The following relations hold:
  \begin{enumerate}
    \item $\ker(A') = \ker(AA')$.
    \item $\col(A) = \col(AA')$.
    \item\label{it:1} $\rank(A) = \rank(AA') = \rank(A'A)$.
    \item $\col(AB)\subseteq\col(A)$.
    \item $\rank(AB)\leqslant\min(\rank(A), \rank(B))$.
    \item $\rank(AB) = \rank(A)$ if A has full row rank.
    \item\label{it:0} $\rank(AB) = \rank(B)$ if B has full column rank.
  \end{enumerate}
\end{lmm}

%\begin{proof}
%
%  \begin{enumerate}
%    \item $\forall x\in\ker A'$, $A'x=0$. Then $AA'x=0$, so $x\in\ker(AA')$. If $x\in\ker(AA')$ then $AA'x=0$, hence $(A'x)'(A'x) = x'AA'x = 0$ gives $A'x=0$; $x\in\ker(A')$.
%    \item $ $
%      
%  \end{enumerate}
%\end{proof}

\begin{lmm}\label{lmm:1}
  Let $A$ be a positive semidefinite $n\times n$ matrix of rank $r$, then an $n\times r$ matrix $T$ exists such that $A = TT'.$ If $A$ is positive definite, then $T$ is $n\times n$ and nonsingular.
\end{lmm}

\begin{proof}
  From semidefiniteness of $A$ we have $S'AS=\Lambda$ with $S'S=I_n$ and diagonal $\Lambda$. Let $\Lambda_1$ be the diagonal $r\times r$ submatrix of $\Lambda$ with positive eigenvalues of $A$. Partitioning $S=(S_1:S_2)$ and from $AS = S\Lambda$, we have $AS_1=S_1\Lambda_1$ and $AS_2=O$ with $S_1'S_1=I_r$, $S_2'S_2=I_{n-r}$, and $S_1'S_2=O$. Note that $A=S\Lambda S'=S_1\Lambda_1 S_1'$. Now let $T=S_1\Lambda_1^{1/2}$, then $TT'=S_1\Lambda_1 S_1'=A$. 
\end{proof}

\begin{thm}\label{thm:0}
  Let $A$ be a positive semidefinite $n\times n$ matrix and $B$ be an $n\times m$ matrix.
  \begin{enumerate}
    \item $\rank(B'AB) = \rank(AB)$.
    \item\label{it:2} If $A$ is positive definite, then $\rank(B'AB) = \rank(B)$.
  \end{enumerate}
\end{thm}

\begin{proof}
  From lemma \ref{lmm:1} we have $A = TT'$ for some $n\times r$ matrix $T$. Then $B'AB = B'TT'B = (T'B)'(T'B)$. 
  \begin{enumerate}
    \item From lemma \ref{lmm:0}\eqref{it:1}, $\rank(T'B) = \rank\left( (T'B)'(T'B) \right)$. So $\rank(T'B) = \rank(B'TT'B)\leqslant\rank(TT'B)\leqslant\rank(T'B)$, which gives $\rank(B'AB)=\rank(T'B)=\rank(TT'B)=\rank(AB)$.
    \item Now $T$ is $n\times n$ and nonsingular, $\rank(B'AB) = \rank(T'B) = \rank(B)$ from lemma \ref{lmm:0}\eqref{it:0}.
  \end{enumerate}
\end{proof}

Note that in theorem \ref{thm:0}\eqref{it:2}, the conclusion does not hold if $A$ is merely invertible. For example, let 
$$B=\begin{pmatrix}0 & 0 & 0\\ 1 & 1 & 1\\0 & 0 & 1\end{pmatrix},\quad A=\begin{pmatrix}1 & -1 & 0\\ -1 & -1 & 1\\0 & 1 & -1\end{pmatrix}.$$ 
Then $$B'AB=\begin{pmatrix}-1 & -1 & 0\\ -1 & -1 & 0\\0 & 0 & 0\end{pmatrix},$$ 
$1=\rank(B'AB)\not=\rank(B)=2$.  
