\documentclass[a4paper,12pt]{book}
\usepackage{amsmath,amsfonts,amssymb,amsthm,graphicx,hyperref,booktabs}
\usepackage[mathscr]{eucal}
\usepackage[a4paper,inner=1.5cm,outer=3cm,top=2cm,bottom=3cm,bindingoffset=1cm]{geometry}
\usepackage[numbers,sort&compress]{natbib}

\usepackage{listings} 
\lstset{                 
basicstyle=\ttfamily,   % code in typewriter font
numbers=left,           % where to put the line-numbers
numberstyle=\tiny,      % the size of the fonts used for the line-numbers
numbersep=5pt,          % how far the line-numbers are from the code
frame=single,           % adds a frame around the code
breaklines=true,        % sets automatic line breaking
columns=flexible        % important!
}

% 中文配置
\usepackage{xeCJK}
\setCJKmainfont{cwTeX Q Ming Medium}
\defaultCJKfontfeatures{AutoFakeBold=4,AutoFakeSlant=.4} 
\newCJKfontfamily\Kai{cwTeX Q Kai Medium} 
\newCJKfontfamily\Hei{cwTeX Q Hei Bold}
\newCJKfontfamily\Yuan{cwTeX Q Yuan Medium}
\newCJKfontfamily\Ming{cwTeX Q Ming Medium}
\newCJKfontfamily\Song{cwTeX Q Fangsong Medium}

\theoremstyle{definition}
\newtheorem{lmm}{Lemma}
\newtheorem{thm}{Theorem}
\newtheorem{dfn}{Definition}
\newtheorem{cor}{Corollary}
\newtheorem{prp}{Proposition}
\newtheorem{ex}{Example}
\newtheorem{prob}{Problem}

% Definition of various shortcuts.
\DeclareMathOperator\curl{curl}
\DeclareMathOperator\divv{div}
%\DeclareMathOperator\curl{\nabla\times}
%\DeclareMathOperator\divv{\nabla\cdot}
\DeclareMathOperator\Curlv{{\bf curl}_\Gamma}
\DeclareMathOperator\im{\Im}
\DeclareMathOperator\re{\Re}
\DeclareMathOperator\comp{\sf{K}}
\DeclareMathOperator\img{\EuScript{R}}
\DeclareMathOperator\pv{pv}
\DeclareMathOperator\dom{dom}
\DeclareMathOperator\nul{\EuScript{N}}
\DeclareMathOperator\rank{rk}
\DeclareMathOperator\col{col}
\DeclareMathOperator\range{range}
\DeclareMathOperator\dist{dist}
\newcommand\bdr{\Gamma}
\newcommand\Div{\divv_\Gamma}
\newcommand\Curl{\curl_\Gamma}
\newcommand\lTS{{\bf L}^2_\text{t}(\mathbb{S}^2)}
\newcommand\lTD{{\bf H}^{-\frac{1}{2}}(\Div)}
\newcommand\lTC{{\bf H}^{-\frac{1}{2}}(\Curl)}
\newcommand\lT{L_{2,\text{t}}(\bdr)}
\newcommand\nltwo[1]{\|#1\|_{L_2(\bdr)}}
\newcommand\nltwod[1]{\|#1\|_{\lTD}}
\newcommand\nltwoc[1]{\|#1\|_{\lTC}}
\newcommand\intv[1]{\int_{\Omega}#1\,\text{d}V}
\newcommand\ints[1]{\int_{\bdr}#1\,\text{d}\sigma}
\newcommand\intvv[1]{\int_{\Omega}#1}
\newcommand\intss[1]{\int_{\bdr}#1}
\newcommand\prb[1]{\begin{prob}#1\end{prob}}
\newcommand\sol[1]{\begin{proof}[Solution]#1\end{proof}}
  
% Trick for defining 'phantom' 
\newcommand\relphantom[1]{\mathrel{\phantom{#1}}}

% number within chapter: e.g.(1.1)
\numberwithin{equation}{chapter}
\numberwithin{thm}{chapter}
\numberwithin{prp}{chapter}
\numberwithin{lmm}{chapter}
\numberwithin{dfn}{chapter}
\numberwithin{cor}{chapter}

% principal value integral
\def\Xint#1{\mathchoice
  {\XXint\displaystyle\textstyle{#1}} 
  {\XXint\textstyle\scriptstyle{#1}} 
  {\XXint\scriptstyle\scriptscriptstyle{#1}} 
  {\XXint\scriptscriptstyle\scriptscriptstyle{#1}} 
\!\int}

\def\XXint#1#2#3{{\setbox0=\hbox{$#1{#2#3}{\int}$ }
\vcenter{\hbox{$#2#3$ }}\kern-.56\wd0}}
\def\ddashint{\Xint=}
\def\dashint{\Xint-}

% in tables with \toprule, set 12pt vertical space
\setlength{\abovetopsep}{12pt}

\begin{document}

\title{Mathematical Problems and Solutions}
\author{Chang-ye Tu}
\maketitle

\begin{prob}[Folklore Knowledge]
  Derive the followings.
  \begin{enumerate}
    \item spherical coordinate, volume element, volume/surface area of $n$-ball
    \item gradient/divergence/laplacian/curl in cylindrical/spherical coordinates
    \item scalar/vector Green identities 
    \item $\int_{-\infty}^\infty e^{-x^2}\,dx$
    \item $\int_0^\infty \frac{\sin x}{x}\,dx$
  \end{enumerate}
\end{prob}

\begin{prob}[Theorems]
  Describe the following theorems, their usages and proofs.
  \begin{enumerate}
    \item inverse function theorem
    \item implicit function theorem
    \item partition of unity
    \item Fatou lemma
    \item Lebesgue dominated convergence theorem
    \item Radon-Nikodym theorem
    \item Calderon-Zygmund decomposition
    \item Rellich-Kondrachov theorem
    \item Sobolev embedding theorem
  \end{enumerate}
\end{prob}

\begin{prob}[Inequalities]
  Describe the following inequalities, their usages and proofs.
  \begin{enumerate}
    \item Cauchy-Schwarz
    \item H\"older
    \item Minkowski 
    \item Hardy
    \item Poincar\'e
  \end{enumerate}
\end{prob}

\prb{Let $\mathbb{P}_l$ be the space of monomials of three variables with total degree less or equal than $l$. Then $$\dim \mathbb{P}_l={l+3\choose 3}.$$}

\sol{Use
  \begin{align*}
    \sum_{k=0}^{l}{m+k\choose k} = \sum_{k=0}^{l}\left\{{m+k+1\choose k}-{m+k\choose k-1}\right\} = {m+l+1\choose m+1}
  \end{align*}
  Then
  \begin{align*}
    \dim\mathbb{P}_l = \sum_{k=0}^{l}{2+k\choose k} = {l+3\choose 3}
  \end{align*}
}

\prb{Evaluate $$\int_0^\infty\frac{1-e^{-k x}}{\sqrt{x^3}}\,\text{d}x,\qquad k\geqslant 0.$$}
\sol{Let
$$f(k)=\int_0^\infty\frac{1-e^{-k x}}{\sqrt{x^3}}\,\text{d}x$$
Then
\begin{align*}
  f'(k)&=\int_0^\infty\frac{\text{d}}{\text{d}k}\left(\frac{1-e^{-k x}}{\sqrt{x^3}}\right)\,\text{d}x \\
       &=\int_0^\infty\frac{e^{-k x}}{\sqrt{x}}\,\text{d}x \\
       &=\frac{\sqrt\pi}{\sqrt k}
\end{align*}
Hence
\begin{align*}
  f(k) = 2\sqrt{k \pi} + C
\end{align*}
From $f(0)=0$ we have $C=0$, so $f(k)=2\sqrt{k \pi}$.
}

\prb{Let $z=\cos\theta$ and $w(z)=y(\theta)$, rewrite the equation
\begin{align*}
  \sin\theta(\sin\theta\,y'(\theta))' - (m^2 + \lambda\sin^2\theta)y(\theta) = 0
\end{align*}
into
\begin{align*}
  (1-z^2) w''(z) - 2z w'(z) - \left(\frac{m^2}{1-z^2} + \lambda\right) w(z) = 0
\end{align*}
}

\sol{
\begin{align*}
 \frac{\text{d}y}{\text{d}\theta}&=\frac{\text{d}y}{\text{d}z}\frac{\text{d}z}{\text{d}\theta} = -y'(1-z^2) \\
 \frac{\text{d}^2 y}{\text{d}\theta^2}&=\frac{\text{d}}{\text{d}\theta}\left(\frac{\text{d}y}{\text{d}z}\right)\frac{\text{d}z}{\text{d}\theta} + \frac{\text{d}y}{\text{d}z}\frac{\text{d}}{\text{d}\theta}\left(\frac{\text{d}z}{\text{d}\theta}\right) \\
 &=\left(\frac{\text{d}}{\text{d}z}\left(\frac{\text{d}y}{\text{d}z}\right)\frac{\text{d}z}{\text{d}\theta}\right)\frac{\text{d} z}{\text{d}\theta} + \frac{\text{d}y}{\text{d}z}\frac{\text{d}}{\text{d}\theta}\left(\frac{\text{d}z}{\text{d}\theta}\right) \\
 &= y''\left(\frac{\text{d} z}{\text{d}\theta}\right)^2 + y' \frac{\text{d}^2 z}{\text{d}\theta^2} \\
 &= (1-z^2) y'' - z\,y'
\end{align*}
Substitute into the equation, we have
\begin{align*}
  -(1-z^2)z\,y'+(1-z^2)\left((1-z^2) y'' - z\,y'\right) - (m^2 + \lambda(1-z^2)) y = 0
\end{align*}
}


\prob{
Let $u\in H^1(\Omega)$. Then $\forall\lambda>0$, $\exists c>0$ s.t.
\begin{align*}
  \intvv{|\nabla u|^2} + \lambda\intss{u^2} \geqslant c\|u\|_{H^1(\Omega)}
\end{align*}
}

\sol{
\url{http://math.stackexchange.com/questions/604577/a-question-about-a-sobolev-space-trace-inequality-dont-understand-why-it-is-tr?lq=1}
}

%\section{Trace Operators}
%\begin{thm}[The trace operator or the generalized boundary values]
%  Let $\Omega$ be a $C^2$ bounded open domain in $\mathbb{R}^n$ with boundary $\partial\Omega$. For any $f(x)\in W^{1,2}(\Omega)$ and for almost all $\xi\in\partial\Omega$, the limiting value $\varphi(\xi)=\lim_{x\to\xi}f(x)$ exists if we let $x$ tend to $\xi\in\partial\Omega$ along the normal at the point $\xi\in\partial\Omega$. Moreover this boundary value $\varphi(\xi)$ of $f(x)$ belongs to $L^2(\partial\Omega)$ and satisfies
%  \begin{align*}
%    \|\varphi(\xi)\|_{L^2(\partial\Omega)}\leqslant C\left\{\|f(x)\|_{L^2(\Omega)} + \sum_{j=1}^n\left\|\frac{\partial f}{\partial x_j}\right\|_{L^2(\Omega)}\right\}
%  \end{align*}
%\end{thm}

%\section{Singluar Integral Equations}
%\section{Mathematical Framework}
%
%\begin{thm}
%  The trace mapping $\gamma_n:v\to n\cdot v|_\bdr$, defined on $D(\overline{\Omega})^n$ has a continuous extension, also denoted by $\gamma_n$, from $H(\divv, \bdr)$ onto $H^{-\frac{1}{2}}(\bdr)$. The kernel of this mapping $\ker\gamma_n$ is $H_0(\divv, \bdr)$.
%\end{thm}
%
%\begin{proof}
%  We start from the Green formula
%  \begin{align*}
%    (v, \grad\varphi)+(\divv v, \varphi)=\int_\bdr n\cdot v\varphi\,d\bdr
%  \end{align*}
%\end{proof}
%
%\begin{thm}
%  The trace mapping $\gamma_\tau:v\to n\wedge v|_\bdr$, defined on $D(\overline{\bdr})^n$ has a continuous extension, also denoted by $\gamma_\tau$, from $H(\curl, \bdr)$ onto $H^{-\frac{1}{2}}(\bdr)$. The kernel of this mapping $\ker\gamma_n$ is $H_0(\curl, \Omega)$.
%\end{thm}

\end{document}
