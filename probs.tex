\documentclass[12pt,a4paper]{article} 
\usepackage{amsmath,amssymb,amsthm,graphicx,hyperref}
\usepackage[BoldFont,SlantFont]{xeCJK}  
\setCJKmainfont{cwTeX Q Ming Medium}
\usepackage[inline,shortlabels]{enumitem}

\theoremstyle{definition}
\newtheorem{prob}{問題}
\newcommand\prb[1]{\begin{prob}#1\end{prob}}

\begin{document}
\title{『縱橫天下』問題選集}
\author{陳冠廷、陳冠儒}
\maketitle

\prb{若$a<0$,且$|a|<1$,試比較$\frac{|a-1|}{|a|-1}$與$-1$的大小關係。(1-1-1-8(1))}
\prb{若$-1\leqslant x<2$,求$|x+1| - 2|x| + |x-2|$的最大值。(1-1-1-12(2))}
\prb{若$|x|\leqslant 1$,$|y|\leqslant 1$,求$|y+1| + |2y - x - 4|$的最小值。(1-1-1-13)}
\prb{某人往返甲乙兩地,去與返的時間比為3:2,且平均速率為7.2公里,則回程的平均時速是多少公里?(1-3-1-5)}
\prb{一輛汽車從甲地開往乙地,如果車速提高20\%,可以比預定時間提前1小時到達。如果想要提前兩小時到達,那麼車速應比原速度提高多少\%?(1-3-1-6)}
\prb{甲每小時跑14公里,乙每小時跑11公里。乙比甲多跑了10分鐘,結果比甲少跑了1公里,求乙跑了多少公里?(1-3-1-7)}
\prb{若$x$的方程式$ax+3 = 2x-b$有兩個不同的解,試求$x$的方程式$(a+b)^{2009}x-\frac{ab}{a+b} x = a - b +5$的解。(1-3-2-12(2))}
\prb{若$x$的方程式$|x| = ax+1$同時有一正根與一負根,且$a$是整數,求$a$。(1-3-2-13(2))}
\prb{若$x$的方程式$||x-a|-b|=3$有三個不同的解,求$b$。(1-3-2-14)}
\prb{若$x$的方程式$||x-2|-1|=a$有三個不同的整數解,求$a$。(1-3-2-14)}
\prb{將55分成四個數;分別把第一個數加1,第二個數減1,第三個數乘2,第四個數除以3之後所得的四個新數都相等。則此四數為?(1-3-3-2(1))}
\prb{父親對兒子說:「我在你現在的年紀時,你才3歲;等你到你現在年紀的2倍少3歲時,我已經72歲了。」求父親、兒子現在各幾歲?(1-3-3-5)}
\prb{客車時速60公里,貨車時速45公里;貨車比客車長135公尺。如果兩車在平行的軌道上同向行駛,客車從後追上貨車,兩車交叉的時間是1分30秒。
\begin{enumerate*}[(i)]
  \item 求兩車長度。
  \item 如這兩車在平行軌道上相向而行,兩車交叉的時間是多少?
\end{enumerate*}(1-3-3-7)}
\prb{有一部隊共有官兵1000人,運輸車5輛,每輛車可載50人。現在要將所有人開赴144公里遠的前方據點(1000名官兵中,不含駕駛兵5人)。假設官兵徒步前進每小時5公里,而車輛往返運送均為時速45公里。若不計人員上下車時間及車輛掉頭時間,試求所有人到達據點最少需要多久?(1-3-3-8)}
\prb{某人中午十二點多外出時,看手錶上兩指針的夾角為$55^\circ$, 一點前回家時發現兩指針夾角仍為$55^\circ$,則此人外出幾分鐘?(1-3-3-9)}
\prb{某人下午六點多外出時,看手錶上兩指針的夾角為$110^\circ$, 下午七點前回家時發現兩指針夾角仍為$110^\circ$,則此人外出幾分鐘?(1-3-3-9(1))}
\prb{在三點與四點之間,時鐘上的分針與時針重合的時刻為?(1-3-3-9(2))}
\prb{實驗得出:一塊重148公斤的銅銀合金在水中會減輕$14\frac{2}{3}$公斤。已知21公斤的銀在水中減輕2公斤;9公斤的銅在水中會減輕1公斤,則此塊合金中含銀多少公斤?(1-3-3-12)}
\prb{某收割隊,第一天收割了稻田的一半多2公頃,第二天收割了稻田所剩部份的25\%,第三天收割了剩下的6公頃,求此塊稻田的面積。(1-3-3-13)}
\prb{糖果若干顆,甲取全部的一半少4顆,乙取剩下的一半多2顆,丙再取剩下的一半少1顆,剩下的6顆全部給丁。求原有糖果幾顆?(1-3-3-13)}
\prb{喬丹在某場球賽後,準備將他的獎金依次按照下述的方法分給他的隊友:第一個隊友分100元與所剩獎金的$\frac{1}{10}$;第二個隊友分200元與所剩獎金的$\frac{1}{10}$;第三個隊友分300元和所剩獎金的$\frac{1}{10}$,依此類推。最後發現獎金正好分光,而每個隊友又分得一樣多的錢,問喬丹有多少隊友?(1-3-3-14)}
\prb{有一旅行團乘客搭乘遊覽車,要求每輛遊覽車的乘客人數相等。起初每輛遊覽車乘22人,結果剩下一人沒上車;如果少一輛遊覽車,所有乘客正好能平均分乘至各車上。已知每輛遊覽車最多能載32人,求此旅行團有多少乘客?(1-3-3-15)}
\prb{猴子第一天吃掉桃樹上所有桃子的$\frac{2}{5}$,還扔掉了4個。第二天吃掉的桃子數再加3個就等於第一天所剩桃子數的$\frac{5}{8}$。此時樹上至少還剩多少個桃子?(1-3-3-15)}
\prb{有若干人分錢若干。若增加四人,則每人少分得一元;若減少五人,則每人多分得兩元。求原有多少人,多少錢?(2-1-3-4)}
\prb{有一個小數,整數部份$a$與小數部份$b$都是三位數,$a$與$b$的和是999。今將整數部份$a$與小數部份$b$對調位置,所得的新數是原數的6倍。求原數?(2-1-3-5)}
\prb{某人騎自行車從甲地以每小時12公里的速度下坡後,以每小時9公里的速度走平路到乙地,這樣共用55分鐘。回程時,他先以每小時8公里的速度通過平路後,再以每小時4公里的速度上坡,這樣共用1小時30分。甲乙兩地相距多少公里?(2-1-3-7)}
\prb{甲乙兩車分別自$A,B$兩地相向而行,甲車時速較乙車時速快8公里,這樣經一小時相遇。若甲以原時速的$\frac{3}{4}$,乙以原時速的$\frac{4}{7}$同時相向而行,則經$\frac{3}{2}$小時相遇。$A,B$兩地相距多少公里?(2-1-3-7)}
\prb{某人沿公路等速前進,每隔4分鐘就遇到迎面而來的一輛公車,每隔6分鐘就有一輛公車從背後超越。假使公車速度一定,迎面而來相鄰兩車距離與從背後而來相鄰兩車距離相等。公車每隔多少分鐘發車一次?(2-1-3-8)}
\prb{某項工程若由甲、乙兩公司承包,則$2\frac{2}{5}$天完成,價格180000元;由乙、丙兩公司承包,則$3\frac{3}{4}$天完成,價格150000元;由甲、丙兩公司承包,則$2\frac{6}{7}$天完成,價格160000元。現在此工程打算由一公司單獨承包,且一週之內需完成,由哪家公司承包價格最低?(2-1-3-9)}
\prb{某班同學參加智力測驗,共$a,b,c$三題。答對$a$得20分,答對$b,c$得25分,答錯皆得0分。測驗結果,每個學生至少答對一題;三題全對1人,答對兩題15人;答對$a, b$題共29人,答對$a,c$題共25人,答對$b,c$題共20人。求此班同學平均分數?(2-1-3-9)}
\prb{$A,B,C$三閥門各以彼此不同的定速注水。三個閥門都打開時,注滿水需1小時、打開$A,C$閥門注滿水需$1.5$小時、打開$B,C$閥門注滿水需2小時。求打開$A,B$閥門注滿水需多少小時?(2-1-3-10)}
\prb{兄弟三人分魚。老大把魚平均分給三人,多一條扔掉;接著老二把自己的魚平均分給三人,多一條扔掉;再來老三把自己的魚平均分給三人,多一條扔掉;最後三人把魚全部合起來均分,多一條扔掉。問原來魚至少幾條?(2-1-3-11)}
\prb{甲乙丙三人各有糖果若干顆,互相贈送;先由甲給乙、丙,分別給的糖果數等於乙、丙原來的糖果數;再由乙給甲、丙,分別給的糖果數等於甲、丙原來的糖果數;最後由丙給甲、乙,分別給的糖果數為甲、乙原來的糖果數。這樣三次互送之後,每人均有64顆糖果。求互送前每人各有幾顆糖果?(2-1-3-11)}
\prb{鋼筆每支100元,原子筆每支30元,鉛筆每支5元。用1000元恰可買此三種筆共100支,求三種筆各買幾支?(2-1-3-13)}
\prb{有一片草地,草每天都長出相同的量。放牧24頭牛,則6天吃完草;放牧21頭牛,則8天吃完草;每頭牛每天吃草量都相同,求
\begin{enumerate*}[(i)]
  \item 如果放牧16頭牛,幾天吃完草?
  \item 要使草永遠吃不完,則最多放牧幾頭牛?
\end{enumerate*}
(2-1-3-14)}
\prb{有一水池,泉水持續湧入。若用12台馬達抽乾水池需5小時,10台馬達抽乾水池需7小時。問若想在兩小時內抽乾水池,最少需幾台馬達?(2-1-3-14)}
\prb{甲乙丙丁四人的平均年齡是三十多歲,如果甲的年齡是乙的$\frac{4}{5}$,乙的年齡是丙的$\frac{3}{2}$,丁比甲多一歲,則四人的平均年齡是幾歲?(2-1-3-15)}
\prb{汽車$A,B$分別從甲、乙兩地同時相向出發,相遇後$A$車經2小時到乙地,$B$車經$\frac{9}{8}$小時到甲地。若甲乙地相距210公里,求
\begin{enumerate*}[(i)]
  \item $A,B$車時速。
  \item 兩車從開始出發到相遇共經過多少小時?
\end{enumerate*}
(2-1-3-ex11)
}
\prb{火車從甲地至乙地,若比平均時速每小時增加5公里,則提早2小時到;若比平均時速每小時慢5公里,則晚2小時半到。求
\begin{enumerate*}[(i)]
  \item 火車平均時速。
  \item 甲乙兩地距離。
\end{enumerate*}
(2-1-3-ex13)}
\prb{若$m>2,n>2$,證明$mn>m+n$。(2-5-1-2)}
\prb{若$d-a<c-b<0$,且$d-b=c-a$,試比較$a,b,c,d$的大小關係。(2-5-1-3)}

\prb{若$a_1,a_2,\dots,a_n$為$n$個互不相同的正整數,證明$\frac{1}{a_1^2} + \frac{1}{a_2^2} + \cdots + \frac{1}{a_n^2}<2$。(2-5-1-6)}
\prb{令$a=-\frac{1996}{1997},b=-\frac{1997}{1998},c=-\frac{1998}{1999},d=-\frac{1999}{2000}$, 試比較$a+d$與$b+c$的大小關係。(2-5-1-ex6)}
\prb{四個互不相等的正數$a,b,c,d$,其中$a$最大,$d$最小,且$a:b=c:d$。試比較$a+d$與$b+c$的大小關係。(2-5-1-ex8)}
\prb{若$a<1$,求解$x$的不等式$ax+a-x < 1$。(2-5-2-5(1))}
\prb{已知滿足$x,y$方程式組$x-y=2, cx+y=3$的$x,y$皆為正數,求$c$的範圍。(2-5-2-ex12)}
\prb{若$a$為正整數,使得$3$介於$\frac{a+5}{a}$與$\frac{a+6}{a+1}$之間,求$a$。(2-5-2-ex13)}
\prb{若$x$的不等式$\frac{x}{4}+\frac{1}{2}\geqslant\frac{x}{3}-\frac{a}{6}$與$2x-3\geqslant 0$之整數解為$2,3,4$,求$a$的範圍。(2-5-2-ex15)}
\prb{求解$x$的不等式$1 < |3-2x| \leqslant 3$。(2-5-2-ex18)}
\prb{求解$x$的不等式$|x-1|>|x+1|$。(2-5-2-ex19)}
\prb{某次選舉選票共有15000張,10位候選人中要選出4位,則候選人至少要得到幾張選票才能當選?(2-5-2-ex20)}
\prb{若干學生分住若干房間。若每間住4人,則有20人沒房間住;若每間住8人,則有一間沒有住滿。求學生幾人,房間幾間?(2-5-2-ex24)}
\prb{有一最簡分數,分子分母和為80。將此分數化為小數且捨去小數點後第二位,所得到的值為$0.7$,則此分數為?(2-5-2-ex26)}
\prb{濃度8\%的鹽水510克與濃度16\%的鹽水170克,再加上多少克的鹽,才能使三者混合起來的鹽水濃度介於15\%到20\%之間?(2-5-2-ex28)}
\prb{某公司去年度男、女員工的人數比為$5:2$,且總人數不超過500人。今年度男、女員工各增加相同的人數,結果男、女員工的人數比變為$12:5$,且總人數超過500人。求今年增加的員工人數?(2-5-2-ex30)}
\end{document}
